%%%%%%%%%%%%%%%%%%%%%%%%%%%%%%%%%%%%%%%%%%%%%%%%%%%%%%%%%%%%%%%%%%%%%%%%%%%%%%%%
%2345678901234567890123456789012345678901234567890123456789012345678901234567890
%        1         2         3         4         5         6         7         8
% DOCUMENT CLASS
\documentclass[oneside,12pt]{Classes/RoboticsLaTeX}

% USEFUL PACKAGES
% Commonly-used packages are included by default.
% Refer to section "Book - Useful packages" in the class file "Classes/RoboticsLaTeX.cls" for the complete list.
\usepackage{amsmath}
\usepackage{amsfonts}
\usepackage{algorithm}
\usepackage{algorithmic}
\usepackage{lettrine}
\usepackage{subfig}
\usepackage{csquotes}
\usepackage[T1]{fontenc}
\usepackage[bitstream-charter]{mathdesign}
\usepackage{graphicx,kantlipsum,setspace}
\usepackage[format=plain, font={it, stretch=1}]{caption}
% SPECIAL COMMANDS
% correct bad hyphenation
\hyphenation{op-tical net-works semi-conduc-tor}
%% ignore slightly overfull and underfull boxes
%\hbadness=10000
%\hfuzz=50pt
% declare commonly used operators
\DeclareMathOperator*{\argmax}{argmax}

%\newcounter{fileAlgorithm}[fileAlgorithm]
\makeatletter
\newenvironment{fileAlgorithm}[1][htb]{%
	%\let\c@algorithm\c@fileAlgorithm %different counter for files
	\renewcommand{\ALG@name}{File}% Update algorithm name
	\begin{algorithm}[#1]%
	}{\end{algorithm}}
\makeatother
%\renewcommand{\theHfileAlgorithm}{Supplement.\thefileAlgorithm}
%\renewcommand{\theHalgorithm}{Supplement.\thealgorithm}

% OPTIONS
\hypersetup{colorlinks=true, linkcolor=blue, citecolor=blue}

% HEADER
\ifpdf
    \pdfinfo {/Title (Cooperative Assembly with Mobile Manipulators in an Underwater Mission Scenario)
              /Creator (TeX)
              /Producer (pdfTeX)
              /Author (Davide Torielli)
              /CreationDate (D:20190110182500) %format D:YYYYMMDDhhmmss
              /ModDate (20190110182500)
              /Subject (Cooperative Assembly for Mobile Manipulators in an Underwater Mission Scenario)
              /Keywords (Thesis, Underwater robotics, marine robotics, kinematic control, TPIK, Mobile Manipulators, Cooperative manipulators, cooperative assembly, AUV, UVMS, ROV, Peg-in-hole, force-torque, stereovision)}
    \pdfcatalog {/PageMode (/UseOutlines)
                 /OpenAction (fitbh) }
\fi

\title{Cooperative Assembly with Mobile Manipulators in an Underwater Mission Scenario}

\ifpdf
  \author{\href{mailto:toridebraus@gmail.com}{Davide Torielli}}
  \collegeordept{\href{http://www.dibris.unige.it}{DIBRIS - Department of Computer Science, Bioengineering, Robotics and System Engineering}}
  \university{\href{http://www.unige.it}{University of Genova}}
  \crest{\includegraphics[width=30mm]{logo_unige}}
\else
  \author{Davide Torielli}
  \collegeordept{DIBRIS - Department of Computer Science, Bioengineering, Robotics and System Engineering}
  \university{University of Genova}
  \crest{\includegraphics[width=30mm]{logo_unige}}
\fi

% DECLARATION
% Use the following command to change the declaration text:
%\renewcommand{\submittedtext}{INSERT NEW TEXT HERE}
\degree{Robotics Engineering}
\degreedate{Semptember, 2019}

%%%%%%%%%%%%%%%%%%%%%%%%%%%%%%%%%%%%%%%%%%%%%%%%%%%%%%%%%%%%%%%%%%%%%%%%%%%%%%%%

\begin{document}

% A page with the abstract and running title and author etc may be
% required to be handed in separately. If this is not so, comment
% the following 3 lines:
% \begin{abstractseparate}
%   %%%%%%%%%%%%%%%%%%%%%%%%%%%%%%%%%%%%%%%%%%%%%%%%%%%%%%%%%%%%%%%%%%%%%%%%%%%%%%%%
%2345678901234567890123456789012345678901234567890123456789012345678901234567890
%        1         2         3         4         5         6         7         8
% THESIS ABSTRACT

% Use the following style if the abstract is long:
%\begin{abstractslong}
%\end{abstractslong}

\begin{abstracts}

  
\end{abstracts}

% \end{abstractseparate}

\maketitle

% add an empty page after title page
\newpage\null\thispagestyle{empty}\newpage

% set the number of sectioning levels that get number and appear in the contents
\setcounter{secnumdepth}{3}
\setcounter{tocdepth}{3}

\frontmatter
%%%%%%%%%%%%%%%%%%%%%%%%%%%%%%%%%%%%%%%%%%%%%%%%%%%%%%%%%%%%%%%%%%%%%%%%%%%%%%%%
%2345678901234567890123456789012345678901234567890123456789012345678901234567890
%        1         2         3         4         5         6         7         8
% THESIS DEDICATION

\begin{dedication}


\begin{center}
	\thispagestyle{empty}
	\vspace*{\fill}
		\begin{flushright}
		``Choose a job you love,\\
		and you will never have to work a day in your life'' \\
		\textit{Confucius (\href{https://quoteinvestigator.com/2014/09/02/job-love/}{maybe})}
		
		\end{flushright}
	\vspace*{\fill}
\end{center}

 

\end{dedication}

% ----------------------------------------------------------------------

%%% Local Variables: 
%%% mode: latex
%%% TeX-master: "../thesis"
%%% End: 

%%%%%%%%%%%%%%%%%%%%%%%%%%%%%%%%%%%%%%%%%%%%%%%%%%%%%%%%%%%%%%%%%%%%%%%%%%%%%%%%
%2345678901234567890123456789012345678901234567890123456789012345678901234567890
%        1         2         3         4         5         6         7         8
% THESIS ACKNOWLEDGEMENTS

% Use the following style if the acknowledgements are long:
%\begin{acknowledgementslong}
%\end{acknowledgmentslong}

\begin{acknowledgements}


Don't forget to acknowledge your supervisor!


\end{acknowledgements}

%%%%%%%%%%%%%%%%%%%%%%%%%%%%%%%%%%%%%%%%%%%%%%%%%%%%%%%%%%%%%%%%%%%%%%%%%%%%%%%%
%2345678901234567890123456789012345678901234567890123456789012345678901234567890
%        1         2         3         4         5         6         7         8
% THESIS ABSTRACT

% Use the following style if the abstract is long:
%\begin{abstractslong}
%\end{abstractslong}

\begin{abstracts}

  
\end{abstracts}


\tableofcontents
\listoffigures
%\printglossary  % Print the nomenclature (WAY TOO COMPLEX FOR ME NOW!)
%\addcontentsline{toc}{chapter}{Nomenclature}

\mainmatter
%%%%%%%%%%%%%%%%%%%%%%%%%%%%%%%%%%%%%%%%%%%%%%%%%%%%%%%%%%%%%%%%%%%%%%%%%%%%%%%%
%2345678901234567890123456789012345678901234567890123456789012345678901234567890
%        1         2         3         4         5         6         7         8
% THESIS INTRODUCTION


\chapter{Introduction}
\label{chap:introduction}
\ifpdf
    \graphicspath{{Introduction/Figures/PNG/}{Introduction/Figures/PDF/}{Introduction/Figures/}}
\else
    \graphicspath{{Introduction/Figures/EPS/}{Introduction/Figures/}}
\fi

Nowadays, Robotics is spreading in any considerable area of human life. Fields such as exploration of deep space and sea, healthcare, industrial application, transportation, show more and more usage of robotics systems. The research has the responsibility to tackle the increasing needs that these many application have.\\
The underwater environment is one of the field where Robotics is in a fast escalation, being the sea so important, from industry to environmental issues. Nowadays, different kinds of robot are largely used for underwater missions. A particular type of submarine robot is the Underwater Vehicle Manipulator System (UVMS): an autonomous underwater vehicle (AUV) capable of accomplishing tasks that require a certain level of dexterity, thanks to single or multiple arms.\\
A really innovative field for underwater missions is the analysis of cooperation between multiple agents, which permits to extend their flexibility. Cooperative robots are two or more robots, identical or different, that coordinate themselves to accomplish various objectives, from mapping an area to assembling an object.   

This thesis focuses on a totally unexplored environment: cooperative \textit{peg-in-hole} assembly with underwater manipulators. It is part of the TWINBOT project \mbox{[\cite{TWINBOT2019}]}, which is devoted to make a step forward for missions in complex scenarios. Effort is devoted to extend the capability of robots to be able to solve more strategic missions.

Part of this thesis is developed at \href{http://www.irs.uji.es/}{IRSLab} at Universitad Jaume I, Castell\`{o}n de la Plana, Spain, during the time I spent as an Erasmus+ 2018/19 student, under the supervision of Professor Pedro J. Sanz and Professor Ra\'{u}l Mar\'{i}n Prades. The IRSLab is the coordinator of the cited TWINBOT project. \\
Another part of the work is done at \href{http://www.graal.dibris.unige.it/}{GRAAL} at Universit\`{a} degli Studi di Genova, Italy, under the supervision of Professor Giuseppe Casalino and Professor Enrico Simetti.\\

The architecture is implemented in C++ and the code is avaible on GitHub at the following link:  \url{https://github.com/torydebra/AUV-Coop-Assembly}. A video of the final experiment is visible at the following link: \url{https://streamable.com/kvoxq} (online; accessed 10-08-2019).\\

This thesis is structured as follows. This Chapter \ref{chap:introduction} introduces the problem, recaps the previous works in this field, and states the objectives of the whole thesis. \mbox{Chapter \ref{chap:control}} defines the theory behind the work, focusing on the Task Priority Inverse Kinematic approach. Chapter \ref{chap:method} applies the theory explained to the actual problem, introducing new methods for the stated problem. In Chapter \ref{chap:results} the simulation set-up is described and results are discussed. Chapter \ref{chap:vision} focuses on the vision part, explaining methods and examining results. In Chapter \ref{chap:conclusions}, conclusions and possible further works are given. In Appendix \ref{chap:AppendixCode} further explanation about the code are given, and in Appendix \ref{chap:AppendixVision} discarded methods (but maybe useful for other purposes) for Vision are briefly presented.

\section{State of the Art}
\label{sec:stateArt}
Robots have been massively introduced in various fields to help humans in different tasks. Nowadays, there are various strategies to use them in underwater environments. A manipulator (robot arm) is considered to be the most suitable tool for executing sub-sea intervention operations. Hence, unmanned underwater vehicles (UUVs), such as remotely operated vehicles (ROVs) and autonomous underwater vehicles (AUVs), are equipped with one or more underwater manipulators.

During the last 20 years, underwater manipulators have been used for many different sub-sea tasks in various fields, for example, underwater archaeology [\cite{IntroApp4}; \cite{IntroApp3}], marine geology [\cite{IntroApp1}; \cite{IntroApp2}], and military applications [\cite{IntroApp5}].

There are many specific tasks where the underwater manipulators are important, from salvage of sunken objects [\cite{IntroSpecApp1}] to mine disposal [\cite{IntroSpecApp2}; \cite{IntroSpecApp3}]. One particular scenario is towards the oil and gas industry, where underwater manipulators are used for pipe inspection, opening and closing valves, drilling, rope cutting [\cite{IntroSpecApp4}], and, in general, to reduced field maintenance and development costs [\cite{IntroSpecApp5}]. A recent survey explored the market related to this technology for oil and gas industry [\cite{IntroSpecApp6}].

\subsection{Previous Works in Underwater Missions}
Since early '90s, research in marine robotics started focusing on the development of underwater vehicle manipulator systems (UVMS). ROVs have been largely used, but they have high operational costs. This is due to the need for expensive support vessels and highly qualified man power effort. In addition, the pilot which operates the vehicle and the arm experiences heavy fatigue in order to carry out the manipulation task.\\


%first works
For the above reasons, the research started to increase the effort toward augmenting the autonomy level in underwater manipulation. For example, to reduce the operator's fatigue, some autonomous control features were implemented in work class ROVs [\cite{IntroTeleopRov}]. Another solution is to completely replace ROVs with autonomous underwater vehicles (AUVs).

Some pioneering projects in this field carried out in the '90s. The AMADEUS project [\cite{IntroAMADEUS1}] developed grippers for underwater manipulation [\cite{IntroAMADEUS2}] and studied the problem of dual arm manipulation [\cite{IntroAMADEUS3}].

The UNION project [\cite{IntroUNION}] was the first to perform a mechatronic assembly of an autonomous UVMS.\\


% dopo
Early 2000s showed many field demonstrations. The SWIMMER project [\cite{IntroSwimmer1}] developed a prototype autonomous vehicle to deploy a ROV mounted on an AUV. This permitted to remove the need of long umbilical cables and continuous support by vessels on sea surface.

This work was followed by the ALIVE project [\cite{IntroAlive1}; \cite{IntroAlive2}], that achieved autonomous docking of Intervention-AUV (I-AUV) into to a sub-sea structure not specifically created for AUV use.

The SAUVIM [\cite{IntroSauvim1}; \cite{IntroSauvim2}] project carried out the first autonomous floating underwater intervention. It focused on the searching and recovering of an object whose position was roughly known a priori. Here, the AUV weighted 6 tons, and the arm only 65 kg, so the dynamics of the two subsystems were practically decoupled and the two controllers were separated. The mission consisted in the AUV performing station keeping while the arm was recovering the object.

After SAUVIM, a project called RAUVI [\cite{IntroRauvi}] took a step further. Here, the AUV performed a hook-based recovery in a water tank. Again, the control of the vehicle and the arm was separated, even if the Girona 500 light AUV and the small 4-degrees-of-freedom (DOF) arm used had more similar masses than the ones used in SAUVIM.

A milestone was the TRIDENT project [\cite{IntroTrident1}]. For the first time, the vehicle and the arm were controlled in a coordinated manner [\cite{IntroTrident2}] to recovery a black-box mockup [\cite{IntroTrident4}]. The used task priority solution dealt with both equality and inequality control objectives, although the inequality ones were only-scalar, except for the joint limits. This permitted to perform some manipulation tasks \textit{while} considering also other objectives, for example, keeping the object centred in the camera frame. In this project, only the kinematic control layer (and not also a suitable dynamic one) was implemented.

The PANDORA project [\cite{IntroPandora1}] focused on increasing the autonomy of the robot, by recognizing failures and responding to them. The work combined machine learning techniques [\cite{IntroPandora2}] and a task priority kinematic control approach [\cite{IntroPandora3}]. However it dealt with only equality control objectives, with a specific ad-hoc solution to manage the joint limit inequality task.

The TRIDENT concepts were enhanced within the MARIS project [\cite{IntroMaris0}]. The used task priority framework [\cite{IntroMaris1}] permitted to \textit{activate} and \textit{deactivate} equality/inequality control objectives of any dimension (not only scalar ones). This project also extended the problem to cooperative agents [\cite{IntroMaris2}].\\


%actual projects
TRIDENT and MARIS concentrated on using the control framework to perform not only grasping actions. A recent work  [\cite{IntroRecent}] analyses how the method can be used in different scenarios, like pipeline inspection and deep sea mining exploration.

The DexROV project [\cite{IntroDexrov}] is studying latencies problems which arise de-localizing on the shore the manned support to ROV operations.

The PROMETEO project [\cite{IntroPrometeo}] plans to improve the use of underwater robotics in archaeological sites. It investigates the manipulation capacity when occlusions of objects can occur and with a wireless communication system to use the robot without umbilical cable.

The ROBUST project [\cite{IntroRobust}] aims to explore and to map deep water mining sites, through the fusion of two technologies: laser-based in-situ element-analysing, and AUV techniques for sea bed 3D mapping.


\subsection{The Control Framework}
\label{subsec:ControlFramework}
In the '90s, industrial robotics researches focused on how to specify control objectives of a robotic system. This was done especially for redundant systems, i.e.\, systems with more degree of freedoms (DOFs) than necessary. This surplus is useful to perform multiple, parallel tasks; for example, avoiding an obstacle with the whole arm while the end-effector is reaching a goal. Given that such systems need to complete different goals, it has become important to have a simple and effective way to specify the control objectives.\\


The task-based control [\cite{IntroTpik1}], also known as operational space control [\cite{IntroTpik2}], defined the control objectives in a coordinate system that is directly linked to the tasks that need to be performed. This idea was followed by the concept of task priority [\cite{IntroTpik3}]. In this theory, a more important task is executed together with a less important task. To accomplish the whole action, the secondary task is attempted only in the null space of the primary one. This means that the secondary task is executed \textit{only if} it does not go against the accomplishment of the first.

This concept was later generalized to multiple task priority levels [\cite{IntroTpik4}]. These works putted the position control of the end-effector as the highest prioritized one, while safety tasks (like joint limits) were only \textit{attempted} at lower priority level.\\


First studies in control of redundant manipulators [\cite{IntroTpik6}; \cite{IntroTpik5}] managed the free residual DOF in such a way to solve the problem of singularity and obstacle avoidance for an industrial manipulator. Another work [\cite{IntroTpik7}] introduced the use of potential functions in industrial manipulators and mobile robots.

A different solution [\cite{IntroTpik8}] proposed a suboptimal approach. The secondary task was solved as if it was alone, but after it was projected in the null space of the higher priority one. To deal with singularities, a variable damping factor was used [\cite{IntroTpik1}]. This solution was later enhanced and called \textit{null-space-based behavioural control} [\cite{IntroTpik9}].
The approach does not deal with the problem of algorithmic singularities that can occur due to rank loss caused by the projection matrix. Further works [\cite{IntroTpik11}; \cite{IntroTpik10}] focused on this problem.\\


Since those times, the task priority framework has been applied to numerous robotic systems, other than redundant industrial manipulators. Some examples includes mobile manipulators [\cite{IntroTpik12}; \cite{IntroTpik13}; \cite{IntroTpik14}], multiple coordinated manipulators [\cite{IntroTpik15}; \cite{IntroTpik16}], modular robots [\cite{IntroTpik17}], and humanoid robots [\cite{IntroTpik18}; \cite{IntroTpik19}]. Furthermore, a stability analysis for several prioritized inverse kinematics algorithms can be found in [\cite{IntroTpik20}].\\


The problem of the classical task priority framework, evident in all the previous mentioned works, is that inequality control objectives (e.g.\ avoiding
joint limits) were never treated as such. In fact, the corresponding tasks were always active, like the equality ones. So, for example, also when the joints are sufficiently far from their limits, the fact that the task is active uselessly adds constraints and \enquote{consumes} DOFs.
Thus, without a transition, the safety control objectives like joint limits could be only considered as secondary. Otherwise, other mission tasks, like reaching a position with the end-effector, can never be accomplished. This led to an undesired situation where safety tasks have a lower priority with respect to non-safety ones.\\


The challenge in activating (inserting) or deactivating (deleting) a task is that these transitions would imply a discontinuity in the null space projector, which leads to a discontinuity in the control law [\cite{IntroTpik21}]. Thus, in the last decade, researches focused on integrate safety inequality control objectives in a more efficient way. 

A new inversion operator was introduced [\cite{IntroTpik22}] for the computation of a smooth inverse with the ability of enabling and disabling tasks in the context of visual servoing. But the work only dealt with the activation and deactivation of the rows of a single multidimensional task (so, not including the concept of different levels of priority). The extension to the case of a hierarchy of tasks with different priorities was provided successively [\cite{IntroTpik23}]. However, the algorithm requires the computation of all the combinations of possible active and inactive tasks, which grows exponentially as the number of tasks increases.

Another work [\cite{IntroTpik21}] modified the reference of each task that was being inserted or being removed, in order to comply with the already present ones, and in such a way to smooth out any discontinuity. However, the algorithm requires $m!$ pseudo-inverses with $m$ number of tasks. For this reason, the authors provided approximate solutions, which are suboptimal whenever more than one task is being activated or deactivated.

Another approach [\cite{IntroTpik25}] directly incorporated the inequality control objectives as inequality constraints in a Quadratic Program (QP). According to this, the idea was generalized to any number of priority levels [\cite{IntroTpik26}]. At each priority level, the algorithm solves a QP problem, finding the optimal solution (in a least-squares sense). Slack variables are used to incorporate inequality constraints in the minimization process. If the solution contains a slack variable different from zero, it will mean that the corresponding inequality constraint is not satisfied. Otherwise, the inequality constraints are propagated to the next level and transformed into an equality ones (to prevent lower priority tasks from changing the best least-square trade-off found). A similar process is done for the equality constraints. A drawbacks of this approach is that the cascade of QP problems can grow  in dimension rapidly. Another issue is that the activation and deactivation of tasks are not considered. This last point is important when temporal sequences of tasks are used, for example when  assembling objects [\cite{IntroTpik28}; \cite{IntroTpik27}].

Instead of a cascade of QP problems, another research [\cite{IntroTpik29}] proposed to solve a single problem finding the active set of all the constraints at the same time. Due to its iterative nature, the authors proposed to limit the number of iterations to achieve a boundary on the computation time, to be more suitable for a real-time implementation. But this solution is not optimal, and, again, activation/deactivation of equality tasks is not considered.

Improvements are made in the already cited TRIDENT project [\cite{IntroTrident1}; \cite{IntroTrident4}], where field trials proved how to consider activation and deactivation of scalar tasks. But the solution still lacks the ability to deal with activation/deactivation of multidimensional tasks, i.e.\ multiple scalar tasks at the same priority level.

%sunto di paper da cui ho copiato
The goals reached by TRIDENT are improved in the MARIS project [\cite{IntroMaris0}; \cite{IntroTpik30}], where, among the other accomplishments, task transitions were successfully implemented in the framework. In particular [\cite{IntroMaris1}], possible discontinuities that could arise are eliminated by a task-oriented regularization and a singular value oriented regularization. Plus, the original simplicity of the task priority framework is retained thanks to pseudo-inverses.


\subsection{The Peg-in-Hole Assembly Problem}
\label{sec:artPeg}
The peg-in-hole is an essential task in assembly processes in various fields, such as manufacturing lines.

This task can be performed following the classical position control method. But this is possible only if precise position of the hole is provided, and the position control error of the robot is zero.
In practice, these conditions can only be obtained in specialized scenario. In the case of more versatile robots, such as underwater manipulators, 
imprecisions and errors are unavoidable.

To deal with these problems, classical works exploit two kind of instruments: cameras and sensors. 
With camera(s), the robot can roughly recognize the objective (i.e.\ the hole) and inspect the overall process. Past researches [\cite{IntroPeg2}; \cite{IntroPeg1}] use this idea to extract boundaries of the object. Another one [\cite{IntroPeg3}] uses visual feedback for a micro-peg-in-hole task (hole of $100 \mu m$).

Other approaches perform precise assembly of the parts thanks to force/torque sensors installed on the wrist. A study [\cite{IntroPeg4}] successfully accomplishes the assembly detecting the force of contact to compensate the positional uncertainty. Newman \textit{et al.} study [\cite{IntroPeg7}] shows how sensors can be used to build map of force and torque values of each contact point.  
In another works [\cite{IntroPeg9}; \cite{IntroPeg8}], the location of the hole was estimated using the measured reaction moment occurred by the contact.
Another good aspect of the sensors  is that they can guarantee stable contact through real-time contact force feedback [\cite{IntroPeg6}; \cite{IntroPeg5}].

Other proposals [\cite{IntroPeg10}; \cite{IntroPeg11}] try to estimate the state of the contact using joint position sensor. This permits to not use the force/torque sensors on the wrist, which would need high control frequency, and would increase overall cost and operation time. 
Some researchers [\cite{IntroPeg12}] show that assembly task can be accomplished without contact force information and with inaccurate vision data. The proposed strategy  mimics the human behaviour: the peg was rubbed in a point close to the object until the relevant objects mated using compliant characteristics. The compliance allows the robot to softly adapt to the hole [\cite{IntroPeg13}; \cite{IntroPeg14}].
A similar, unexpensive, approach is tested experimentally [\cite{IntroPeg15}], without the use of force/torque sensors (i.e.\ no force feedback), nor Remote-center-compliance devices, and with inaccurate hole information.




\section{Motivation and Rationale}
Sea plays an important role in our societies. Many examples are given in the previous section \ref{sec:stateArt}. When such a kind of environment are so important, exploitation of robotic systems is necessary at different level.\\
This thesis aims to improve the current state of the art in underwater intervention missions. Efforts in this direction can help the robots to accomplish more and more complicated tasks, substituting gradually their remotely operated versions (ROVs), and, in some cases, humans. This would help to reduce the cost, to increase the safety, to boost the performances, and, in general, to accomplish missions that before were unthinkable.

The peg-in-hole is one example of these complicated tasks. In general, robotic assembly problems have been addressed and explored widely, but, to the best of this author's knowledge, no works have been done for cooperatively assembly in underwater scenarios, except for the TWINBOT project [\cite{TWINBOT2019}], which this thesis is part of. Productions related to this problem can help to fill this current lack and can make the technology to advance. 

Cooperative agents augment the capability of the single, for example to carry an heavy object. It is important to notice that cooperation here is intended at \textit{kinematic level}. So, for example, we are not speaking about robotic swarms, where more \enquote{planning} cooperation is explored with Artificial Intelligence techniques. Instead, here cooperation means two (or, in general, more than two) robots that share (in some way) their commanded system velocities (the usual output of the kinematic layer) to move together, without, in this case, make the common tool fall or break. There is not \enquote{high level} reasoning with a planner, but there are mathematical formulas with vectors and matrices to make the robot \textit{cooperate}.\\
Such improvements at kinematic level are important because they reduce the work-effort at higher levels (i.e. the planner), and they make the overall system faster.\\

At the time this thesis was being developed, the TWINBOT project was in an early stage. So, this works evolves autonomously, always keeping an eye on the main objective of the project: improving capabilities of cooperative underwater intervention robots. The aim  is to developed a kinematic control framework suitable for the cooperative underwater \textit{peg-in-hole} problem stated, considering methods that can be used also for other robotics missions.\\
The Task Priority Inverse Kinematic (TPIK) approach is exploited for the single agent kinematic control (section \ref{sec:tpik}), for the arm-vehicle coordination (section \ref{sec:armVehScheme}), and for the cooperation scheme (section \ref{sec:coopScheme}).\\
A force-torque sensor is used to have data from the collisions that happen between the \textit{peg} and the \textit{hole} during the insertion phase. This information is used by two different methods which help to achieve the final goal and to reduce frictional forces that can damage the objects or the robots.\\
The first one is a new objective called Force-Torque objective (section \ref{sec:forceTask}). Its duty is to reduce the forces and the torques that act on the peg, moving it into the opposite direction. The objective is inserted into the TPIK list, among the other objectives. This is noticeable because we exploit a \enquote{dynamic} information at kinematic level.\\
The second method is called Change Goal routine (section \ref{sec:changeGoal}). Its job is to change the origin of the goal frame (that is inside the hole) where the control drives the peg. This is done to compensate possible errors in the hole's pose. For example, if the goal is slightly on the left respect to the exact centre of the hole, lot of collisions happen with the left inner side, so lot of forces directed on the right are detected. This routine simply moves the origin of the goal frame according to the detected forces, on the right in the cited example. 
However, it must be considered that the \textit{peg-in-hole} is simplified: the problem of having too big errors, which makes the peg to collide with the external face of the hole, is not considered.

Another part of the thesis focuses on computer vision, specifically on the pose estimation of the hole (Chapter \ref{chap:vision}). In particular, stereo-vision methods are used by a third robot, which acts exclusively for the Vision purpose.\\
The Vision routine is divided into two phases.\\
The first is the \textit{detection} (section \ref{sec:visDetect}) where the hole's structure is found in the images that the camera captures. Among all the algorithms tried, one based on shape detection and another based on template matching are discussed in the final results (section \ref{subsec:detectResult}).\\
The second phase, the \textit{tracking} (section \ref{sec:visTracking}), uses the first one as initialization for a \textit{markerless model-based} method. In this case, trials with a mono-camera, a stereo-camera, and depth stereo-camera are compared and discussed.
 

%todo vedi se inserire sta cosa come altra subsec di state of art
%%% A Task Priority Approach to Cooperative Mobile
%%% Manipulation: Theory and Experiments
%\subsection{Cooperative manipulators}
%Cooperative robots indicates robots that \textit{cooperate} at some level to accomplish certain task. 
%%This thesis focus on cooperative mobile manipulators: in general, two robots that carry a common load with their arms.\\
%
%First works on cooperative mobile manipulation [2,3] involved a mobile base dynamically coordinated with the arm using a potential function. The aim was maintaining the manipulator joint positions close to their midrange.\\
%In another old work [4], a team of mobile robots push an object
%cooperatively toward a desired position. Another works [5,6] follows this idea and focused on motion planning. It divides the problem in a global path planner and a local manipulation one.\\
%In [10] a centralized robust adaptive controller is used. This was to guarantee that the motion and force trajectories of the constrained object converge to the desired manifolds.
%However, the approach is limited to a 3 DOFs system and
%neglects all the other control objectives of the system.\\
%An important issue related to mobile manipulators is that disturbances of the mobile platform (such as imprecise wheel actuation, or underwater currents) would propagate through the coupled kinematics to the end-effector. This happen if the whole body Jacobian is employed to control the end-effector. A solution [12], similar to the previous cited one [2], consists in decouple the manipulator motion and the mobile platform motion kinematically in task space. The tests done on two dual-arm manipulators shown that disturbances are
%efficiently compensated before propagating to end-effector
%level. However, no other objectives are taken into account. Same authors investigates the model for the internal forces and dynamics interactions [13, 14].\\
%
%The control strategies are classified usually in two types: the centralized control paradigm and the decentralized 
%control paradigm (Khatib, et al., 1996). In the centralized paradigm, there is a central controller which coordinates the 
%robots (Chen and Li, 2006). This type of controller is relatively easy to be designed, but is difficult to be implemented because the great amount of numeric calculations and communication of dates to be transmitted to the robots. In the decentralized paradigm, each robot has an individual controller. which makes this type more practical, although communication between the controllers is typically necessary. The classical way is the leader-follower approach(Hirata, et al., 2003),
%studied also for non-manipulator cooperative  transportations (i.e.\ robots without arm) [8, 7, 9].
% 
%Recent works in cooperative robotics include the already cited project in underwater robotics MARIS [\cite{IntroMaris2}; 18) but also aerial manipulators[20].



 



%%%%%%%%%%%%%%%%%%%%%%%%%%%%%%%%%%%%%%%%%%%%%%%%%%%%%%%%%%%%%%%%%%%%%%%%%%%%%%%%
%2345678901234567890123456789012345678901234567890123456789012345678901234567890
%        1         2         3         4         5         6         7         8
% THESIS CHAPTER


\chapter{Control Framework}
\label{chap:control}
\ifpdf
    \graphicspath{{ControlFramework/Figures/PNG/}{ControlFramework/Figures/PDF/}{ControlFramework/Figures/}}
\else
    \graphicspath{{ControlFramework/Figures/EPS/}{ControlFramework/Figures/}}
\fi

\section*{Introduction}
In this section, the control framework implemented is discussed. The architecture is constituted by two parts:
\begin{itemize}
	\item The Mission Manager, which job is to supervision the execution of the overall mission. It provides the \textit{action}, a list of control objective that the Kinematic Control Layer must satisfy.
	\item The Kinematic Control Layer (KCL) focus on provide the system velocities (i.e. vehicle and joint velocities), given the list of control objectives from the Mission Manager.
\end{itemize}

This architecture is build from the ones used in \cite{IntroMaris2}, \cite{tesiWander}, \cite{IntroRecent}. The sections \ref{sec:definitions}, \ref{sec:controlObjectives}, \ref{sec:tpik}, \ref{sec:armVehScheme}, and \ref{sec:coopScheme} derive from these works and are here recalled.

\section{Definitions}
\label{sec:definitions}
In this section, principal used notations are described.
\begin{itemize}
	\item The system configuration vector of the robot $ \boldsymbol{c} \in \mathbb{R}^{n}$, ~
	$\boldsymbol{c} \triangleq 
		\begin{bmatrix}
			{\boldsymbol{q}} \\ \boldsymbol{\eta}
		\end{bmatrix}$,\\
	where $\boldsymbol{q} \in \mathbb{R}^{l}$ is the l-DOF arm configuration vector and  $\boldsymbol{\eta} \in \mathbb{R}^6$ is the vehicle \emph{generalized coordinate position vector}. The first three components of $\boldsymbol{\eta}$ are the position vector $\boldsymbol{\eta}_1 \triangleq \begin{bmatrix}x \\ y \\ z\end{bmatrix}$, with components in the inertial frame $\langle w \rangle$. The last three components of $\eta$ are the orientation vector $\boldsymbol{\eta}_2 \triangleq \begin{bmatrix}\phi \\ \theta \\ \psi\end{bmatrix}$ expressed in terms of the three angles roll, pitch, yaw (applied in the yaw-pitch-roll sequence \cite{fossenAnglesSeq}). The singularity given by this Euler sequence that arise when $ \theta = \pi/2$ is handled by a specific control objective (i.e. \textit{horizontal attitude}).
	(TODO) %todo linka sez) 
	From the explained definition, it results that $ n = l+6 $
	
	\item The system velocity vector or the robot $\dot{\boldsymbol{y}} \in \mathbb{R}^n$, ~
	$\dot{\boldsymbol{y}} \triangleq 
	\begin{bmatrix}\dot{\boldsymbol{q}} \\ \boldsymbol{v}\end{bmatrix}$,\\
	where $\dot{\boldsymbol{q}} \in \mathbb{R}^{l}$ are the arm joint velocities and $\boldsymbol{v} \in \mathbb{R}^{6}$ is the vehicle velocity vector. The first three component of $\boldsymbol{v}$ are the linear velocities $\boldsymbol{v}_1 \triangleq \begin{bmatrix}\dot{x} \\ \dot{y} \\ \dot{z}\end{bmatrix}$ and the last three are the angular velocities $\boldsymbol{v}_2 \triangleq \begin{bmatrix}p \\ q \\ r\end{bmatrix}$, both with components in the vehicle frame $\langle v \rangle$. The vehicle is considered fully actuated, so the vector $\dot{\boldsymbol{y}}$ concincides with the control vector used by the kinematic layer.
\end{itemize}
	
\section{Control Objectives}
\label{sec:controlObjectives}
Let us consider what the robot need to achieve. An \textit{objective} is a job that the robot must accomplish during the mission. Different objectives can be requested at the same time, for example we want the joints to stay away from their physical limits, the robot to maintain an horizontal attitude, and the end-effector to reach a desired pose.
\subsection{Control Objectives classification}
\label{sec:coClass}
Control objectives can be classified in order of importance in the execution of the mission. Some of them can be more important of others. For example, usually we prefer that the robot do not hurt humans over the reaching of a goal. Another example could be that the robot should first act to not damage itself while accomplish the mission. In general this give the idea that the more important objectives have to been satisfied first, and then, \textit{if possible}, also the less important ones. 
A general classification based on the \textit{priority} is given here (from the more important class to the less important one):
\begin{itemize}
	\item \textit{Physical Constraints} objectives. This include interaction with environment (e.g not push against a rigid surface, impose a cooperative tool velocity).
	\item \textit{System Safety} objectives, e.g. avoiding joint limits or obstacles.
	\item \textit{Prerequisite} objectives. This is for objectives needed to accomplish the mission, like focus the camera on the object to be grasped
	\item \textit{Mission} objectives, the actual objective that define the mission, like reach an end-effector position.
	\item \textit{Optimization} objectives, to choose, among the possible solution (if multiple ones exist) the best one. To example, to choose the one which has the best energy efficiency.  
\end{itemize}

\subsection{Equality and Inequality Objectives}
We consider a variable  $ \boldsymbol{x}(\boldsymbol{c}) \in \mathbb{R}^m $, dependent on the robot configuration vector $ \boldsymbol{c}$, with $ p $ the control objective \textit{dimension}. The control objective can be of two types:
\begin{itemize}
	\item \textit{Equality control objective}, the requirement, for $t \to \infty$, that \\ \mbox{$\boldsymbol{x}(\boldsymbol{c}) = \boldsymbol{x}_0$}.
	
	\item \textit{Inequality control objective}, the requirement, for $t \to \infty$, that \\ \mbox{$\boldsymbol{x}(\boldsymbol{c}) < \boldsymbol{x}_{max}$ ~ or ~ $\boldsymbol{x}(\boldsymbol{c}) > \boldsymbol{x}_{min}$ ~or ~$ \boldsymbol{x}_{min} < \boldsymbol{x}(\boldsymbol{c}) < \boldsymbol{x}_{max}$}.
\end{itemize}
Please note that here symbols $= , < , >$ refers to element-by-element comparison of the vectors.\\

\subsection{Reactive and non Reactive Control Task}
\label{sec:reactNonReact}
For each control objective, there is always an associated \textit{feedback reference rate}. The aim is to drive the variable $\boldsymbol{x}(\boldsymbol{c})$ toward a point $ \boldsymbol{x}^* $ where the control objective requisite is satisfied. The used example of \textit{feedback reference rate} is:
\begin{equation}
	\boldsymbol{\dot{{\bar{x}}}} (\boldsymbol{x}) \triangleq \gamma (\boldsymbol{x}^* - \boldsymbol{x}),\quad \gamma > 0
\end{equation}
That is a simple proportional law, where $\gamma$ is a positive gain proportional to the desired convergence rate for the considered variable.\\

To link the considered variable $ \boldsymbol{x}(\boldsymbol{c})$ to the system velocity vector, the following relationship is used:
\begin{equation}
\label{eq:CartJacVel}
	\dot{\boldsymbol{x}} = \boldsymbol{J} \dot{\boldsymbol{y}}
\end{equation} 
that express how system velocity vector $\dot{\boldsymbol{y}}$ influences the rate of change of the variable. $ \boldsymbol{J} \in \mathbb{R}^{m \times n}$ is the so-called \textit{task-induced Jacobian}.\\
Having the actual $\dot{\boldsymbol{x}}$ as much as possible equal to the desired reference $\boldsymbol{\bar{x}}$ is called \textit{reactive control task}.\\

There are situation where a task has not an associated control objective. For example, it happens when an external command (e.g. an human operator, or an imposed vehicle velocity) provide directly the reference velocities. In this case, there is no desired $\boldsymbol{x}^*$ to reach, and the reference is generated by something else. It this case, we speak about \textit{non-reactive control task}.

%As an example, we take the objective is to drive the end-effector in a desired pose, with only the arm and considering a fixed vehicle. This problem is well know as \textit{inverse kinematic} problem; i.e. finding the joint velocities $\dot{\boldsymbol{q}} \in \mathbb{R}^l$ which bring the pose $\boldsymbol{x} \in \mathbb{R}^6$ in a specific one. In this case we have:
%\begin{equation}
%\dot{\boldsymbol{x}} = \begin{bmatrix}
%	\underset{3\times3}{\boldsymbol{J_{k,0}}} & \underset{3\times3}{\boldsymbol{0}} \\  \underset{3\times3}{\boldsymbol{0}} & \underset{3\times3}{^{w}\boldsymbol{R}_v}
%	\end{bmatrix}	
%	\dot{\boldsymbol{q}}	
%\end{equation}

%todo ACTIONS? se le uso...

\subsection{Control Objectives Activation and Deactivation}
\label{sec:activations}
During the execution of a mission, not always each inequality control control objective is relevant. As an example, maintaining joints away from their mechanical limits is a safety task which is needed only when the joints are actually near its limits. When a joint is sufficiently far away, there is no necessity to overconstrain the system imposing a velocity. To deal with this, we speak about \textit{activation} and \textit{deactivation} of control objectives and their relative control task.
Let us define the following \textit{activation function}:
\begin{equation}
	a(x) \in [0,1]
\end{equation}
as a continuous, sigmoidal, function, which assumes $0$ value outside the validity region of the control objective, and $1$ inside it. In between, a smooth transition is present, to gently activate/deactivate the control objective. \\
For example, considering a scalar ($ p = 1 $) inequality control objective with the requirement $x(\boldsymbol{c}) > x_{min}$ the \textit{activation function} is defined as:
\begin{equation}
	\label{eq_activation_f}
	a(x) \triangleq
	\begin{cases}1,& x(\boldsymbol{c}) < x_{min}\\
	s(x),& x_{min} \leq x(\boldsymbol{c}) \leq x_{min} + \Delta\\
	0, & x(\boldsymbol{c}) > x_{min} + \Delta\\
	\end{cases}
\end{equation}    
where $s(x)$ is a smooth decreasing function joining the two extreme value $1$ and $0$, and $\Delta$ a value to create a zone where the inequality is satisfied but the activation is between $1$ and $0$ to prevent chattering problems. Similar definition can be done for the other two kind of inequality control objectives.\\

In general, when multidimensional control objectives ($m > 1$) are present, the activation takes the form of a diagonal matrix:
\begin{equation}
\boldsymbol{A} \triangleq
	\begin{bmatrix}
	a_1 & & \\
	& \ddots & \\
	& & a_m	 
	\end{bmatrix}
\end{equation}

Obviously, for equality control tasks the activation is not defined, because they are always \enquote{active}.\\
For \textit{non-reactive} control tasks, the activation is simply $\boldsymbol{A} \equiv 	\begin{bmatrix}
1 & & \\
& \ddots & \\
& & 1	 
\end{bmatrix}$, being absent the variable $x(\boldsymbol{c})$


\section{Task Priority Inverse Kinematics}
\label{sec:tpik}
We describe an \textit{Action} $\mathcal{A}$ as list of prioritized control objectives. Each objectives is positioned at a defined priority level $k$. With this notation, the following symbols are defined:
\begin{itemize}
	\item $\dot{\bar{\boldsymbol{x}}}_k \triangleq \begin{bmatrix}\dot{\bar{x}}_{1,k} & \cdots & \dot{\bar{x}}_{k_m,k}\end{bmatrix}^T$ is the vector of the reference velocities for the control task $k$, of dimension $k_m$.
	\item $\dot{\boldsymbol{x}}_k \triangleq \begin{bmatrix}\dot{x}_{1,k} & \cdots & \dot{x}_{k_m,k}\end{bmatrix}^T$ is the current rate of change of the $k$ task.
	\item $\boldsymbol{J}_k$ is the Jacobian relationship which relates the current rate-of-change $\dot{\boldsymbol{x}}_k$ with the system velocity vector $\dot{\boldsymbol{y}}$ as in equation \eqref{eq:CartJacVel}.
	\item $\boldsymbol{A}_k \triangleq \textrm{diag}(a_{1,k},  \cdots,  a_{k_m,k})$ is the diagonal matrix of all the activation functions described in section \ref{sec:activations}
\end{itemize}
It is important to notice that different objectives can have same priorities $k$. In this case, it is possible to simply stack the vectors and matrices to obtain a objective and a related task $k$ that includes both objectives. Without loss of generality, different objectives will be considered always at different priority levels.

The aim of the kinematic layer is to find the system velocity vector $\boldsymbol{\bar{y}}$ that satisfies \textit{as much as possible} the requirements of each objective of the action $\mathcal{A}$. Given the presence of different objectives with different priorities, it must be taken into account to satisfy higher priority objective first. To do this, a sequence of nested minimization problems must be solved: 
\begin{equation}\label{eq:rminproblem}
S_k \triangleq \left\{ \arg \mathrm{R\textrm{-}}\min_{\dot{\bar{\boldsymbol{y}}} \in S_{k-1}} \left\| \boldsymbol{A}_k (\dot{\bar{\boldsymbol{x}}}_k - \boldsymbol{J}_k \dot{\bar{\boldsymbol{y}}}) \right\|^2 \right\},\, k = 1, 2, \ldots, N,
\end{equation}
where $S_0 \triangleq \mathbb{R}^n$, $S_{k-1}$ is the manifold of solutions of all the previous tasks in the hierarchy, and $N$ is the total number of priority levels. This is the so called \textit{Task Priority Inverse Kinematic} (TPIK). The notation $\mathrm{R\textrm{-}}\min$ is introduced in \cite{IntroMaris1}. The so called \textit{iCAT} (inequality Constraints And Task transitions) framework solve the \eqref{eq:rminproblem} with the algorithm \ref{alg:icat}.\\
\begin{algorithm}[H]
	\caption{iCAT} \label{alg:icat}
	\begin{algorithmic} [1] {\large
		\STATE{$\boldsymbol{\rho}_{0} = \boldsymbol{0}$}
		\STATE{$\boldsymbol{Q}_{0} = \boldsymbol{I}$}
		\vspace{5px}
		\FOR{k=1 \TO N}
		\vspace{5px}
		\STATE{$\boldsymbol{W}_k =  \boldsymbol{J}_k \boldsymbol{Q}_{k-1} (\boldsymbol{J}_k \boldsymbol{Q}_{k-1})^{\#,\boldsymbol{A}_k,\boldsymbol{Q}_{k-1}}$}
				\vspace{5px}
		\STATE{$\boldsymbol{Q}_k = \boldsymbol{Q}_{k-1} (\boldsymbol{I} - (\boldsymbol{J}_k \boldsymbol{Q}_{k-1})^{\#,\boldsymbol{A}_k,\boldsymbol{I}} {\boldsymbol{J}_k \boldsymbol{Q}_{k-1}})$}
		\vspace{5px}
		\STATE{$\boldsymbol{\rho}_k = \boldsymbol{\rho}_{k-1} + \textrm{Sat}\left( \boldsymbol{Q}_{k-1} (\boldsymbol{J}_k \boldsymbol{Q}_{k-1})^{\#,\boldsymbol{A}_k,\boldsymbol{I}} \boldsymbol{W}_k \left(\dot{\bar{\boldsymbol{x}}}_k - \boldsymbol{J}_k \boldsymbol{\rho}_{k-1} \right) \right)$}
		\vspace{5px}
		\ENDFOR 
		\vspace{5px}
		\STATE{$\dot{\boldsymbol{\bar{y}}} = \boldsymbol{\rho}_N$}}
	\end{algorithmic}
\end{algorithm}

\noindent The special pseudo inverse operator $\boldsymbol{(\cdot)}^{\#,\boldsymbol{A},\boldsymbol{Q}}$ [\cite{IntroMaris1}] manages some invariance problems of \eqref{eq:rminproblem}; the function Sat($\cdot$) [\cite{antoSat}] controls the variable saturations. Details of the procedure can be found again in [\cite{IntroMaris1}].

\subsection{Notes on Conflicting Objectives}
From section \ref{sec:coClass}, it should be understood that lower priority task are not always satisfied. The problem with this arise when the main mission objective, that is not at the higher priority, can't be never accomplished, thus failing the general mission. This can be the case with an obstacle: the robot may stuck in a point of \textit{local minima} (that is anyway better than crash into it). This is a general problem of all reactive controlling method. The solution must be found at the mission manager level, which should plan another path or another sequence of Actions. This problem is not considered in this work. 

%todo
%%\subsection{Notes on underactuation} %todo ? vedi un po


\section{Arm-Vehicle Coordination Scheme}
\label{sec:armVehScheme}
\begin{figure}[H]
	\begin{center}
		\includegraphics[width=0.90\columnwidth]{vehArmCoordScheme}
		\caption[Arm-Vehicle Coordination Scheme in the TPIK]{A scheme showing the two Task Priority Inverse Kinematics blocks for the arm-vehicle coordination implementation}\label{fig:veharmcoord}
	\end{center}
\end{figure}
Inaccuracy in velocity tracking for vehicle can have effects on the arm. 
A relevant problem arises when disturbances of the floating base, caused by thrusters or its large inertia, propagate and affect the end effector motions [\cite{IntroMaris2}].
To solve this, a kinematic decoupling of arm and base is done, implementing it within the task priority approach The idea is to have two parallel TPIK as shown in \ref{fig:veharmcoord}:
\begin{itemize}
	\item The first TPIK 1, given the Action $\mathcal{A}$ consider the vehicle together with the arm as a whole full controllable system. From its output $\dot{\boldsymbol{\bar{y}}}$ only the vehicle reference velocity are taken.
	\item The second TPIK 2 consider the vehicle as totally non controllable. So, a \textit{non-reactive} task (\ref{sec:reactNonReact})  is used at the top of the priority list to \textit{constrain} the output velocities of the vehicles to the actual one. From the total output $\dot{\boldsymbol{\bar{y}}}$, only the manipulator part is taken.\\
	In this way, the manipulator velocity are \textit{optimized} to follow \textit{at best} the objectives of the action $\mathcal{A}$ considering the \textit{measured} vehicle velocity and their influence on the objectives.	
\end{itemize}

In general, a multi-rate control of arm and vehicle is used, which means that velocities for arm and vehicle are given at different frequency. This schema is suitable for such an implementation: the TPIK 2 can run at higher frequency, updating the manipulator commanded velocities more frequently that the vehicle commanded ones.

\section{Cooperation Scheme}
\label{sec:coopScheme}
\begin{figure}[H]
	\begin{center}
		\includegraphics[width=0.60\columnwidth]{coopFrames.jpg}
		\caption{The frames of the two cooperative vehicles carrying a common object}\label{fig:coopFrames}
	\end{center}
\end{figure}

The discussion about cooperation is here explained, limiting it to include only two cooperating robotic systems. This is used to transporting a common object. The coordination policy described take care of the bandwidth restriction in underwater scenario. Thus, it deals with the cooperation in a decentralized manner, limiting the exchange of information.\\
It is assumed that the object is held firmly by both agents, so no sliding happens during the missions. The two robots agree on a shared fixed frame, so, their respective tool frames $\langle t_a \rangle$ and $\langle t_b \rangle$ and the object frame $\langle o \rangle$ are coincident: 
\begin{equation*} 
\langle t \rangle \triangleq \langle t_a \rangle = \langle t_b \rangle = \langle o \rangle
\end{equation*}
In the figure \ref{fig:coopFrames}) the cited frames are shown.
The firm grasp assumption imposes that
\begin{equation}\label{eq:coopintro}
	\dot{\boldsymbol{x}}_t = \boldsymbol{J}_{t,a} \dot{\boldsymbol{y}}_a = \boldsymbol{J}_{t,b} \dot{\boldsymbol{y}}_b
\end{equation}
with $\dot{\boldsymbol{x}}_t$ the object velocity with component on $\langle t \rangle$, $\dot{\boldsymbol{y}}_a$ and $\dot{\boldsymbol{y}}_b$ the system velocity vectors of agents $a$ and $b$ as described in section \ref{sec:definitions}, and $\boldsymbol{J}_{t,a}$ the system Jacobians of agents $a$ and $b$ with respect to $\langle t \rangle$. These Jacobians tells how the tool velocities $\dot{\boldsymbol{x}}_t$ are affected by system velocities $\dot{\boldsymbol{y}}_a$ and $\dot{\boldsymbol{y}}_b$. Due to the firm grasp assumptions, the tool velocities generates by $\dot{\boldsymbol{y}}_a$ and $\dot{\boldsymbol{y}}_b$ must be equal.\\

%Let us rewrite the equation \eqref{eq:coopintro} as:
%\begin{equation}\label{eq:coopintro2}
%\begin{gathered}
%\begin{bmatrix}
%\boldsymbol{J}_{t,a} & -\boldsymbol{J}_{t,b}
%\end{bmatrix}
%\begin{bmatrix}
%\dot{\boldsymbol{y}}_a \\ \dot{\boldsymbol{y}}_b
%\end{bmatrix}
%\triangleq \boldsymbol{G}\dot{\boldsymbol{y}}_{ab}=0 \quad \Longleftrightarrow \quad \dot{\boldsymbol{y}}_{ab} \in ker(\boldsymbol{G}),
%\end{gathered}
%\end{equation}
%This represents the subspace where $\dot{\boldsymbol{y}}_{ab}$ is constrained to lay for the firm grasp assumption.\\

The equation \eqref{eq:coopintro} derived from the grasp constrain can be expressed in the Cartesian space as:
\begin{equation}
	\dot{\boldsymbol{x}}_t = \boldsymbol{J}_{t,a} \boldsymbol{J}^\#_{t,a} \dot{\boldsymbol{x}}_t =  \boldsymbol{J}_{t,b} \boldsymbol{J}^\#_{t,b} 
	\dot{\boldsymbol{x}}_t 
\end{equation}
\begin{equation}
\label{eq:constrainMatrixC}
	(\boldsymbol{J}_{t,a} \boldsymbol{J}^\#_{t,a} - \boldsymbol{J}_{t,b} \boldsymbol{J}^\#_{t,b}) 
	\dot{\boldsymbol{x}}_t \triangleq \boldsymbol{C} \dot{\boldsymbol{x}}_t = \boldsymbol{0}
\end{equation}
\begin{equation}
	\dot{\boldsymbol{x}}_t \in ker(\boldsymbol{C}) = Span(\boldsymbol{I} - \boldsymbol{C}^\#\boldsymbol{C})
\end{equation}
The kernel of $\boldsymbol{C}$, called \textit{Cartesian Constraint Matrix}, express the space of achievable object velocities at the current configuration.\\
The idea of the scheme is to put a non-reactive task at the top of the priority, to constrain the desired object velocity $\dot{\boldsymbol{\tilde{x}}}$ in this subspace. In this way both agents can follow this desired object velocity despite the different situation caused by other objectives.\\
\begin{figure}[H]
	\begin{center}
		\includegraphics[width=1\columnwidth]{coopScheme.png}
		\caption{The cooperation algorithm with its different steps }\label{fig:coopScheme}
	\end{center}
\end{figure}
The algorithm, schematized in fig. \ref{fig:coopScheme} proceeds as follows:
\begin{itemize}
	\item In the first step, the two agents run the algorithm of \ref{sec:tpik} as they were the only to act. So, we have:
	\begin{equation}
		\dot{\boldsymbol{x}}_{t,a} = \boldsymbol{J}_{t,a} \dot{\boldsymbol{y}}_{a} , \qquad 
		\dot{\boldsymbol{x}}_{t,b} = \boldsymbol{J}_{t,a} \dot{\boldsymbol{y}}_{b}	
	\end{equation}
	where, in general, the two \textit{non cooperative} tool velocities are different: $\dot{\boldsymbol{x}}_{t,a} \neq \dot{\boldsymbol{x}}_{t,b}$.
	
	\item The tool velocities are exchanged (i.e. sent to the coordinator) and a \textit{cooperative} tool velocity is computed:
	\begin{equation}\label{eq:weightsum}
		\dot{\hat{\boldsymbol{x}}}_t = \dfrac{1}{\mu_a + \mu_b} (\mu_a \dot{\boldsymbol{x}}_{t,a}  + \mu_b \dot{\boldsymbol{x}}_{t,b}), \qquad
		\mu_a , \mu_b > 0	
	\end{equation}
    \begin{equation}
		\begin{gathered}
			\mu_a = \mu_0 + \| \dot{\bar{\boldsymbol{x}}}_t - \dot{\boldsymbol{x}}_{t,a} \| \triangleq \mu_0 + \| \boldsymbol{e}_a \|, \\
			\mu_b = \mu_0 + \| \dot{\bar{\boldsymbol{x}}}_t - \dot{\boldsymbol{x}}_{t,b} \| \triangleq \mu_0 + \| \boldsymbol{e}_b \|, \\
			\mu_0 > 0
	    \end{gathered}
	\end{equation}
	where $\dot{\boldsymbol{\bar{x}}}_t$ is the ideal velocity that, if applied, would asymptotically take the tool to the desired goal.\\
	The \textit{cooperative} tool velocity is a \textit{weighted} compromise between the two \textit{non cooperative} ones. The \textit{weights} $\mu_a, \mu_b$ given more freedom to the robot which meet the highest error $\boldsymbol{e}$. This error is a way to understand how much one robot is in difficult in tracking the \textit{ideal} tool velocity $\dot{\boldsymbol{\bar{x}}}_t$.
	
	\item The new \textit{cooperative} tool velocity $\dot{\hat{\boldsymbol{x}}}_t$ is not, in general, a \textit{feasible} velocity that both vehicle can perform. So, an additional passage is required:
	\begin{equation}
		\dot{\tilde{\boldsymbol{x}}}_t \triangleq \big( \boldsymbol{I} - \boldsymbol{C}^\# \boldsymbol{C} \big) \dot{\hat{\boldsymbol{x}}}_t
	\end{equation}
	with $\boldsymbol{C}$ defined in \eqref{eq:constrainMatrixC}.
	
	\item Each agent run a new TPIK procedure, identical to the first one, but with a \textit{non-reactive} control objectives to track the \textit{feasible cooperative} velocity $\dot{\tilde{\boldsymbol{x}}}_t$. The output of the two agents algorithm, $\dot{\boldsymbol{\hat{y}}}_a$ and $\dot{\boldsymbol{\hat{y}}}_b$ will be the final velocity which the kinematic layer provide.\\
	Moving the equality control objective to make the end effector reach the goal at the top of the hierarchy does not influence the safety tasks. This property is proven in \cite{tesiWander}.
\end{itemize}

In this described method, the only information that the agent must exchange are the
\textit{non-cooperative} velocities $\dot{\boldsymbol{x}}_{t,a}, \dot{\boldsymbol{x}}_{t,b}$, the feasible ones $\dot{\boldsymbol{\tilde{x}}}$, and the constrain matrix $\boldsymbol{C}$. Furthermore, even less data can be exchanged if the \textit{coordinator} is a procedure which runs on a robot, and it is not on an external node. 






%%%%%%%%%%%%%%%%%%%%%%%%%%%%%%%%%%%%%%%%%%%%%%%%%%%%%%%%%%%%%%%%%%%%%%%%%%%%%%%%
%2345678901234567890123456789012345678901234567890123456789012345678901234567890
%        1         2         3         4         5         6         7         8
% THESIS CHAPTER


\chapter[Control Architecture: Methods \& Results]{Control Architecture: \\ Methods \& Results}
\label{chap:method}
\ifpdf
    \graphicspath{{Method/Figures/PNG/}{Method/Figures/PDF/}{Method/Figures/}}
\else
    \graphicspath{{Method/Figures/EPS/}{Method/Figures/}}
\fi

\begin{figure}[H]
	\centering
	\includegraphics[width=14.5cm]{scenario_whole.png}
	\caption[The Scenario with the two robot carrying the peg and the vision robot watching the hole]{The Scenario of the experiment. The two twin robots are carrying the common peg, while the third robot is watching the hole to estimate its pose.}
	\label{fig:method_uwsim}
\end{figure}

In this chapter, experimental set-up is described, and results are given and discussed. The code for the whole architecture is available at \href{https://github.com/torydebra/AUV-Coop-Assembly}{https://github.com/torydebra/AUV-Coop-Assembly}.\\
%todo and explained a bit in the appendix?

The scenario is made up of two I-AUV's \href{https://cirs.udg.edu/auvs-technology/auvs/girona-500-auv/}{Girona 500 AUV} underwater vehicles, each one equipped with a CSIP Robot arm5E (4 DOF arm with a parallel yaw gripper). The final goal is to successfully coordinate the robots in such a way that the peg, hold by both manipulators, is inserted correctly in the other piece, fixed in the environment. One robot is equipped with a force torque sensor that permits to understand forces applied on the peg, caused during the insertion phase. This information is provided to both robots.

The chosen strategy divides the problem in two phases: Hole Detection and Insertion. In the first preliminary step are done to detect the hole. A third robot, not used for manipulation task, is in charge to exploit vision to estimate the pose of the hole. Detail about this are given in section \ref{chap:vision}.
The second phase explores the problems inherent to the interaction between the peg and the hole, and the communication between the carrying agents. This is described in this chapter.
                          
%todo descrivi la situa: due piu uno robot, il peg, hole nel ambiete. e metti foto per mostrare ste cose

%todo dire in sect 1 blabla... ecc

\section{Simulators}
\label{sec:simulators}
A bit effort has been spent to choice a suitable simulator for the case. At the end, \href{http://www.irs.uji.es/uwsim/}{UWSim} [\cite{uwsim}] was chosen. It is a simulator largely used for this kind of scenarios, which visualize a virtual underwater scene. It provides a different variety of useful sensors (e.g. the used force-torque sensor and the cameras), but also others can be added. It is fully integrate in ROS, which made it really easy to use. ROS is used as simulator interface: through ROS messages, we can send commands to the robots and we can receive information from the going on test. Contact physics is implemented integrating the physics engine \href{https://pybullet.org/wordpress/}{Bullet} with \href{http://www.openscenegraph.org/}{OSG} through \href{https://github.com/mccdo/osgbullet}{OSGBullet}. To further details about how the simulation is implemented, especially the contact physics part, please refers to the documentation of the cited software. The cons in using UWSim is that the simulations is fully kinematic, so no dynamics interaction ar present. This means, for example, that velocity sent to the robot are immediately accomplished, and that we can't simulate the real physics while grasping a real object. For the scope of this thesis, this lack is not important because dynamics is not considered. Furthermore, the only needed dynamic part, i.e. how the contact between the tool and the hole affect the whole manipulator chain, can be simulated at kinematic level thanks to the information provided by the force-torque sensor, as explained in section \ref{sec:forceConsideration}.\\
To fill the lack of dynamic of UWSim, a good alternative can be \href{https://github.com/freefloating-gazebo/freefloating_gazebo}{FreeFloatingGazebo} [\cite{freeFloatingGazebo}]. In truth, this simulator is a plug-in for Gazebo and UWSim; it integrates them in order to achieve both dynamic (thanks to Gazebo) and visually realistic I-AUV simulation (thanks to UWSim). The interface used to communicate with the simulation is the same of UWSim, so ROS and its messages, which make it easy as UWSim to use. \\
This plug-in has been taken into consideration for dynamic test, that are not evaluated due to the lack of time, but can be certainly used for further works.
\href{http://gazebosim.org/}{Gazebo} is a generic simulator widely used in all robotics fields. It is the de-facto simulator for ROS. Seen its purpose, it is not a ready-to use simulator for underwater environment, and can be only a starting point to build a software to simulate this particular scenario (as it is done by FreeFloatingGazebo).\\
Also other simulators, \href{http://www.coppeliarobotics.com/index.html}{V-REP} [\cite{vrep}] and \href{https://cyberbotics.com/}{Webots} [\cite{webots}] have been taken into consideration but discarded for same \enquote{not ready-to-use} reason like Gazebo.\\
An interesting simulator is \href{https://github.com/disaster-robotics-proalertas/usv_sim_lsa}{USV simulator} [\cite{usvsim}] which takes the best from UWSim, Gazebo and FreeFloatingGazebo to implement realistic simulation. This is a really recent and in development project, and however it is focused more on surface vessels dynamics.\\

More details and comparisons are available in \cite{simComparisonCook} and \cite{usvsim}, and a schematic recap taken from \cite{usvsim} is visible in fig. \ref{fig:simComparison}.
\begin{figure}[H]
	\centering
	\includegraphics[width=14cm]{simComparison.png}
	\caption[Schematic Simulators Comparison]{Schematic recap of simulation comparison taken from \cite{usvsim}. $\times$ stands for no implemented feature; ~ $\surd\,$  for a feature that is a discrete representation of the real one; ~ $\surd\surd\,$ for a good representation of the real one. More details on how each feature is evaluated are available in the original paper.}
	\label{fig:simComparison}
\end{figure}

\section{Assumptions}
It is important to detail the main assumptions (related to the control architecture) made. A lot of problems, that it is necessary to take into account in a real environment, are not explored. This is necessary due to the difficulty of the particular scenario chosen.
\begin{itemize}
	\item Simulation is only kinematic. This implies, for example, that the commanded velocity to the vehicle and the arm are accomplished \textit{instantaneously} and \textit{perfectly}. Another implication is that the movements of arm and of the vehicle don't influence each other at all. The only \enquote{dynamic} implemented is caused by collision between the peg and the hole, which transfer the forces and torques acting on the peg along the whole arm. Disturbances of this type on the vehicle are neglected. (TODO %todo e invece ce le metterò?%)
	
	\item The initial configuration is with the peg already \textit{correctly} grasped by both robots. Also, the position of the grasped point and the peg dimension are known: this imply that relative position between robot and tip of the peg is \textit{perfectly} known. This initial condition is chosen because the grasping phase and problems arising during cooperative transportation have been explored in others cited project like MARIS and ROBUST (e.g in the work \cite{IntroMaris2}) and also as part of the on-going project TWINBOT.
	
	
	\item Pose (linear and angular displacement) of the two carrying robots and of the vision robot respect a common inertial frame is known. In real situation, knowing the position underwater is really an issue and it is never really precise. This information can be provided, for example, thanks to some mappings of the seafloor, to find a common interest point which refer to. Note that it is not important know the pose of the robot respect to a point above the water surface (that can be done thanks to information shared with surface vessels, for example as explored the WiMUST project [\cite{wimust}]). The important thing is to know relative pose of the robots to a common node, that can also be underwater. This is needed to make the vision robot share correctly the estimated pose of the hole.
	
	\item No real communication issues between the two carrying robot are taken into account. The presence of water put important issues in real situation; for example with water \textit{full-duplex} communication (i.e. sharing data \textit{at the same time}) is impossible, and in general a there is lower bandwidth than in the air. Some experiment in simulated environment with different methods of underwater communication are detailed in \cite{IntroMaris2}.\\
	The issues about communication are taken indirectly using a cooperative scheme (explained in section \ref{sec:coopScheme}) which permits to exchange few information between the two carrying agents.
	
    \item During the insertion phase, the architecture resolve alignment errors \textit{only if} the peg is inside the hole. If the peg touches the external hole surface, no routines are implemented to deal with the problem of \textit{looking} for the hole on the surface. In the test, the pose estimation of the hole is sufficiently good to not cause this problem.
	
	\item The sensor is positioned on the tip of the peg, and provided force and torques respect to this point. This would obviously not possible in real applications. Anyway, no generality is loss due to this, because we could simply project the force torque sensor information on another frame. Plus, with the chosen simulator, the sensor must be positioned on the robot part (i.e. the peg) to detect forces acting on this particular part. Both robot have access to the sensor data without uncertainties (expect error due to how the simulator computes collision).
	
	%todo altre? 
\end{itemize}

Others assumptions, more related to the vision part, are detailed in section \ref{sec:visioAssumption}.



\section{Simulating the Firm Grasp Constrain}
\label{sec:firmGrasp}
As stated, in the experiment there are two carrying robots and one vision robot. Vision robot and its work is described in chapter \ref{chap:vision}.\\
Due to the limitation of the used simulator, \href{http://www.irs.uji.es/uwsim/}{UWSim} [\cite{uwsim}], some tricks have to be made. Without simulation of dynamics, grasping the peg is impossible. The simulator permits to fake the grasp with an \textit{object picker} sensor: when an object is sufficiently near to the point where this sensor is, it becomes \enquote{grasped} and it will rigidly move with the whole robot. The problem is here we have two robot that must grasp, so the object can't rigidly move with both, but only with the first who catch it. Furthermore, external forces applied to an object (grasped or not) can't be detected with the force-torque sensor, because it only detect forces acting on a vehicle part.\\
To solve this, two pegs are modelled as additional link for both robot. In this way, each peg is rigidly attached to its own robot. Now, the problem is how to maintain the two pegs perfectly coincident during the whole mission. In an ideal case, the cooperation algorithm generates robots velocities for each agent in such a way that they cause the \textit{identical} Cartesian velocities to the peg. But Jacobian derives from approximation of non-linear relationship. During the transportation, but especially during the collision propagation (section \ref{sec:forceConsideration})
the two pegs distance themselves a bit. This cause an increasing problem because one of the two robot expect the peg to be in a different position.\\
In real scenario, firm grasp act like a \enquote{glue}: if the end effector tends to go away from the grasping point, friction acts to maintain it to the contact point. This is true for very small errors; if the cooperation performance is bad, the common tool falls down or something brokes.\\
In the simulation, to fake the firm grasp, an additional routine is implemented: it simply calculates the distance between the two peg, and generates robot velocities to zeroing this distance. It is important to note that this is an help that we would have also in real scenario, as explained before. The only difference is that, in real scenario, if the errors are too big the end-effector begins to slips, and it never returns to its original grasping point. In this case, it returns always to the initial point. The velocity generates by this routine are not so big to hide bad cooperation; so the tests are suitable to evaluate the proposed architecture, and to simulate real situation.\\
In conclusion, this produces an additional disturbance $\boldsymbol{\dot{q}}_{\varepsilon}$


\section{Objective Prioritized List}
In this section, the objectives inserted into the TPIK procedure are explained.\\ Objectives related to safe transportation (e.g. obstacle avoidance), grasping (e.g. camera centring object) are not considering because they are out of the focus of the chosen mission, and also because they are explored in other works [\cite{IntroMaris2}; \cite{IntroRecent}].\\

The first TPIK, the one where the two robot acts independently to each other (section \ref{sec:coopScheme}) is:
\begin{itemize}
	\item \textbf{Joint Limits avoidance} (\textit{reactive, inequality, safety}): this objective keep joint away for their mechanical limits. It must be at high priority because it is a safe task, and also must be an inequality objective to not overconstrain the system when joints are away from their limits.
	
	\item \textbf{Horizontal Attitude} (\textit{reactive, inequality, safety}): to maintain the vehicle horizontal respect to the water surface. Most of the underwater vehicle are are passively stable in roll and pitch and these DOF are not controllable, so this objective is only needed for fully actuated vehicles (as stated in this case).
	
	\item \textbf{Tool position control} (\textit{reactive, equality, mission}): this is the objective that define the mission. It is used to bring the tool towards the defined goal (i.e. inside the hole).
	
	\item \textbf{Preferred Arm Shape} (\textit{reactive, inequality, optimization}): this is a low priority objective that, \textit{if possible}, maintain the arm in a predefined shape. This shape permits the arm to have good dexterity but it is also useful to transport the peg in a natural way.
	
	
\end{itemize}
The categories (written in italic) are explained in section \ref{sec:controlObjectives}; further explanations on these and other tasks are available in \cite{IntroMaris2}, \cite{tesiWander}, \cite{IntroRecent}. Please note that in the code there is also an additional \textit{last task} which is used to cancel out any practical discontinuities during task activations [\cite{IntroMaris1}].\\

The TPIK procedure is run two more times, one for the vehicle-arm coordination (section \ref{sec:armVehScheme}), the other for the cooperation between the two robots (section \ref{sec:coopScheme}). Respectively, two \textit{non-reactive} objectives are put at the top of the hierarchy listed above.\\
The final output will be the velocity command $\boldsymbol{\dot{\bar{y}}}$.\\

\noindent Please note that the collision propagation (section \ref{sec:forceConsideration}) and the firm grasp constrain (section \ref{sec:firmGrasp}) produces additional \enquote{disturbances} that will be added to $\boldsymbol{\dot{\bar{y}}}$.
%todo chiedi se ha senso sta frase successiva
In particular, the collision propagation disturbance is added to the control for the first robot, while the disturbances cause by the firm grasp constrain to the second. In this way, collision \textit{directly} affect only the first agent, but also affect the other one \textit{indirectly} because the latter is \textit{dragged} by the firm grasp constrain.\\

%todo ricorda le force consideration della theory part

\section{Results}
\begin{figure}[H]
	\centering
	\includegraphics[width=10cm]{scenario_onlyTwin.png}
	\includegraphics[width=10cm]{scenario_onlyTwin2.png}
	
	\caption[Scenario for test without the vision part]{Two differents angle view of the initial scenario for the results presented in this section. The pose estimation with vision is here neglected and the goal frame considered known without any errors.}
	\label{fig:onlyTwin_uwsim}
\end{figure}


%%%%%%%%%%%%%%%%%%%%%%%%%%%%%%%%%%%%%%%%%%%%%%%%%%%%%%%%%%%%%%%%%%%%%%%%%%%%%%%%
%2345678901234567890123456789012345678901234567890123456789012345678901234567890
%        1         2         3         4         5         6         7         8
% THESIS CHAPTER


\chapter{Vision}
\label{chap:vision}
\ifpdf
    \graphicspath{{Vision/Figures/PNG/}{Vision/Figures/PDF/}{Vision/Figures/}}
\else
    \graphicspath{{Vision/Figures/EPS/}{Vision/Figures/}}
\fi

% short summary of the chapter
\section*{Summary}

%todo foto truzza del robot che guarda con le tre camere sotto

Before the twin robots can approach the hole, its position must be know, at least roughly. In this chapter, the pose estimation of the hole is discussed.\\
In the considered scenario, a third robot is present. Its duty is exclusively to \textit{detect} \& to \textit{track} the hole. In the simulation, another \href{https://cirs.udg.edu/auvs-technology/auvs/girona-500-auv/}{Girona 500 AUV} is used for this job, without the arm. Is evident that, in real scenario, a littler and more efficient robot should be used for the vision, see that no manipulation task are needed. In fact, in the original TWINBOT [\cite{TWINBOT2019}] simulation, a smaller \href{https://bluerobotics.com/product-category/rov/bluerov2/}{BlueROV} is present, as can be seen in (TODO) %todo ref figura original twinbot.
However, in this case, another Girona 500 is used to not deal with another robot model.\\
The \textit{vision} robot is equipped with two identical cameras which point in front of it. They are used as:
\begin{itemize}
	\item Two distinct cameras, independent one of the other.
	\item As stereo cameras, thus exploiting stereo vision algorithms.
	\item As RGB-D camera, i.e., a stereo vision couple where the left one is a RGB camera and the right one a Depth camera.  
\end{itemize} 

For the sake of simplicity, some assumptions are made:
\begin{itemize}
	\item Known \textit{intrinsic} camera parameters. These parameters are used by algorithms to take into account how the single camera see the scene. The (known) distortion is zero.
	\item Known \textit{extrinsic} camera parameters, i.e. the position and orientation of cameras (respect the vehicle), and thus the relative pose (the transformation matrix) from one camera to the other (needed for stereo vision algorithm).
	\item No external disturbances for the images, such as light reflections underwater or bad visibility.
	\item Hole model known. This means that dimensions of the cuboid which contain the hole are known. Further explanation about this are given successively in (TODO) %todo linka sect tracking
	\item A "friendly" cuboid hole. The front face is coloured and additional hole are present, as can be seen in (TODO) %todo linka figura truzza
	. This help both the \textit{detection} phase and the \textit{tracking} phase.
\end{itemize}
About the robot, other assumption are:
\begin{itemize}
	\item the pose of the vehicle respect the inertial frame is known. (TODO?AS explained?? se si linka sez). This permits to know the hole pose estimation respect the inertial frame, to directly send the robot which are carrying the peg.
	\item The initial position of the robot is such that it is facing the front face of the hole. It must be noticed that methods explained in the next sections can be adapted to relax this hypothesis. For example, the robot could turn around z-axis until the hole is detected. (TODO?? %todo maybe far vedere sti good result? )
	 Also, good results are obtained when the robot not exactly face directly the cuboid, but, for example, it is on its side, looking at front face and a side one.
	\item Once the robot has detected the hole and the pose sent to the twin robots, it must go away to not interfere with the insertion phase. This is done through keyboard (as a ROV) but it is not difficult to improve the code to let him go away autonomously. It must be noticed that, thanks to the \textit{tracking} algorithms, if the robot moves (because it is commanded to do so, or for water currents) the pose estimation is still good. 
\end{itemize}

The job is done into two phases: \textit{Detection} \ref{sec:visDetect}
 and \textit{Tracking} \ref{sec:visTracking}

%todo conclusion dicendo che si puo anche detectare il peg...

\section{Detection}
\label{sec:visDetect}

\textit{Object Detection} means detect a particular shape (the \textit{object}) in the scene. This is important to initialize the tracking algorithm. In fact, to track an object, its initial pose (or at least some particular points of it) must be know. Thus, the detection part is the most difficult part.\\
For the tracking algorithm used successively, the detection part must provide a correspondence between some points in the 2D image and some points in the 3D object shape. It is important to notice that the 3D coordinate needed refers to object frame, and not to an "external" frame. Seen that the object model is assumed to be know, the 3D coordinate of some point directly derive from this, so no other assumptions are made. So, in the experiment, a .init file looks like this:
\begin{algorithm*}
	4        \hspace{40px}      \# Number of points \\
	          \hspace*{50px}        \#xyz with x going in, y on the right, z down. Unit measure is meter\\
	0      -0.4     -0.4  ~ \# top right corner\\
	0      0.4      -0.4  ~   \# top left\\
	0      0.4     0.4   ~  \# bottom left\\
	0      -0.4    0.4    ~ \# bottom right\\
\end{algorithm*}\\
Indicating the position on the 4 corners of the front face, respect to a frame positioned in the center of the hole, with x-axis going inside the hole, y lying along the surface pointing on the right, z pointing down to the seafloor.\\
Four points is the minimum number of point accepted by the tracking algorithm. The more the point are, the more the tracking is good. Plus, point should lying on different surfaces of the object, to have better results. Anyway, good tracking result are obtained also not considering these two aspects.\\

The work of Detection is to provide 2D coordinate as $(x,y)$ of these point in the 2D image captured by camera. This must be done for each camera, except for the Depth one, when used.\\
Two method are used: \textit{Find Square Method} and \textit{Template Matching}. A third method, in which the 2D positions of the points are chosen by human operator, is used to have a benchmark of the other two and to analyse the tracking results when 2D Coordinates are almost perfect.\\
Other methods and functions are briefly explained in Appendix \ref{chap:AppendixVision} 

\subsection{Already known Coordinate Method}
As explained, with this method the 2D coordinate are perfectly known. The code do this by letting the user to click on the 4 pixel which contains the corners. It is assumed that the user click on the best pixels which contain the corners. Given that the image is made by discrete pixels, is impossible to have an ideal point which is exactly the corner, but the errors for this are not noticeable.\\

\subsection{Find Square Method}
This method is taken from an OpenCV \href{https://docs.opencv.org/3.4/db/d00/samples_2cpp_2squares_8cpp-example.html}{tutorial}.\\
Details of how each function works and explanation of the computer vision algorithm used are not provided here, to not go outside the scope of the thesis. Only a rough explanation of how the method works is presented:
\begin{itemize}
	\item This function looks in each image channel (unique if is a gray image, three if is a colored image) to find squares.
	\item First, it pre-process the image to reduce noise.
	\item Then, \href{https://docs.opencv.org/3.4.6/d3/dc0/group__imgproc__shape.html#ga17ed9f5d79ae97bd4c7cf18403e1689a}{\textit{findContours()}} is called to retrieves contours of the square with the algorithm described in \cite{findcountors}.
	\item Each contour is approximated to be more like regular polygon, with lesser vertices and edges.
	\item Finally, it look if the shapes are similar to squares/rectangles. This is done checking if the internal angles are almost 90 degree.
	\item The returned squares are described by its four corners, that is what we are looking for. An additional function is called to be sure that the order of the returned corner is the same order of the point described in the .init file, otherwise correspondence are obviously erroneous.	
\end{itemize}

In the way it works, it should be noticeable that this algorithm give good results only if the camera approximately face the cuboid structure at the front. (TODO as can be see se ti metti di lato...). If the side face is more visible, it will be the one detected. So, we must know which side the robot is facing to give the 3D correspondence points in the .init file.\\
In addition, this method is suitable if no other squares of similar dimension are present, otherwise further work is needed to discriminate them. Also, is not suitable with other kind of shapes (a pipe hole, for example).\\
Given the described initial pose, TODO result are good.


\subsection{Template Matching Method}
Differently from the previous one, this method belong to a wider class of well-known method. \textit{Template Matching} means to find a pattern (in this case, the face of the hole) inside a scene. So, an additional image of the square face of the hole is needed.\\
The code developed follow an OpenCV \href{https://docs.opencv.org/3.4.6/de/da9/tutorial_template_matching.html}{tutorial}. As the previous method, details on how template matching works go outside the scope of this thesis.\\
In brief, template matching find the point in the scene which as more similarity (or less dissimilarity) with the provided template. This is done considering intensity values of the pixels in the neighbourhood area of a center pixel. In practice, the template is shifted all over the scene image and some calculation for each new template shifting are done. Various formula are provided by OpenCV and are detailed in the previous linked of the tutorial.\\
It is important that the template is being scaled up and down, because usually its size is not equal to the size of the object in the scene. For each scaling, a best similarity point is detected. Then, all the similarity points are compared and the best one chosen.  At the end, a rectangle with the template dimension (scaled) is build considering this point as the center. The corners of the rectangle are the 4 point which we was looking for.\\

TODO RESULT OF THISSSSSS

This method is less robust than the previous for different initial position of the robot. If the face is view from a different angle, other template image is needed, with an orientation similar to what the robot is seeing. In general, lot of template images at different angles are needed. Also, building the shape around the centre point is more difficult, if this shape is not a square.





\section{Tracking}
\label{sec:visTracking}

%%%%%%%%%%%%%%%%%%%%%%%%%%%%%%%%%%%%%%%%%%%%%%%%%%%%%%%%%%%%%%%%%%%%%%%%%%%%%%%%
%2345678901234567890123456789012345678901234567890123456789012345678901234567890
%        1         2         3         4         5         6         7         8
% THESIS CONCLUSIONS
%\def\baselinestretch{1}
\chapter{Conclusions}
\label{chap:conclusions}
\ifpdf
    \graphicspath{{Conclusions/Figures/PNG/}{Conclusions/Figures/PDF/}{Conclusions/Figures/}}
\else
    \graphicspath{{Conclusions/Figures/EPS/}{Conclusions/Figures/}}
\fi
%\def\baselinestretch{1.66}

This thesis has presented a kinematic control architecture for two cooperative autonomous underwater manipulators.\\
The robot collaboration is done at kinematic level, exchanging vectors and matrices to produce a common tool Cartesian velocity. The cooperation scheme takes into account that underwater communication is difficult, and it keeps the amount of exchanged data as low as possible.\\
The experimental results showed how the Task Priority Inverse Kinematic approach used can deal with an assembly task: the \textit{peg-in-hole}. Being an unexplored problem for cooperative underwater manipulator, the scenario is simulated with many simplifications and assumptions.\\
Part of the problem deals with computer vision techniques. The thesis shows how some detection and tracking algorithm can be exploited to estimate the hole's pose.\\
For the insertion phase, a force-torque sensor is used to help the accomplishment of the mission, thanks to the data provided exploited by the control architecture.\\
Both the Control and the Vision part can help other works in various robotics fields, not only related to underwater intervention missions.\\

Code implementation is suited to easily manage objectives (e.g. to delete offline some objectives to try a different method). Big effort has been made in providing a modular and flexible code architecture. For example, the dependency from ROS (Robotics Operating System) is kept at minimum to make possible to easily adapt the code to a different kind of interface and/or simulator (which does not rely on ROS to communicate).\\


In the Chapter \ref{chap:introduction}, an introduction about the context is given, and the previous works in the relative fields has been recalled.\\
In Chapter \ref{chap:control}, the principal points of the theory behind the Control Architecture are summarized. Here, the mathematical foundations for the Task Priority Inverse Kinematic approach are recalled, considering also a coordination policy between multiple agents that fits in the chosen approach.\\
In Chapter \ref{chap:method}, the theory explained before is exploited to deal with the scenario stated by this thesis. A Force-Torque objective is inserted in the TPIK list to reduce the magnitudes of forces and torques that act on the peg during the insertion phase. This is noticeable because it is used a \enquote{dynamic} information (the force and the torque) in a pure-kinematic method. A simple, but suitable for the scenario, list of objectives is then described. An additional routine, part not of the kinematic layer but more of the Mission Managing layer, is explained. Considering the goal frame where the peg is driven to by the kinematic control, this new routine shifts its origin according to the direction of the forces acting on the peg. This is an additional method to further exploit the information given by the force-torque sensor.\\
In Chapter \ref{chap:results}, the simulated environment is detailed and the experimental results are discussed. The chosen simulator, UWSim, is introduced, along with some other underwater simulators that can be useful for the interested reader. To make the insertion phase realistic, collisions between the peg and the hole have been inserted in the simulation. These collisions propagates through the whole robotic system, thus affecting the arm and the vehicle. Another routine is implemented to fake a firm grasp of the tool by the two robots. In real environment, this constraint is assured by frictional forces, but in a pure-kinematic simulator like UWSim the frictions are not present. The tuning of the gains permits to not hide bad cooperation between the agents.  In the same Chapter \ref{chap:results}, assumptions to simplify the problem are explained, and an idea of how the Control Loop runs is given. Finally, results of the experiments done are presented and discussed. Three main experiments have been carried out: without hole's pose error, with a fixed error of 0.015m along one axis, and one final test which includes the Vision part with the hole's pose error given by the \textit{Detection} and the \textit{Tracking} algorithms used.\\
Chapter \ref{chap:vision} covers exclusively methods and tools used by the Vision robot. No theoretical background is given because it would take out of the scope of this thesis. The Chapter gives an idea on how two computer vision libraries, \href{https://opencv.org/}{\textbf{OpenCV}} (Open Source Computer Vision Library) [\cite{opencv}] and \href{https://visp.inria.fr/}{\textbf{ViSP}} (Visual Servoing Platform) [\cite{visp}], are used. The Vision part is divided into two phases: Detection and Tracking. For both, different algorithms have been tested, compared and discussed. 
In particular, for the Detection part, some methods have been discarded and they have not been used any more for further trials, but they are anyway presented in Appendix \ref{chap:AppendixVision}. They are all OpenCV algorithms that can help in other applications.\\
This Chapter \ref{chap:conclusions} concludes the thesis and it gives some starting points for possible future works.\\
The Appendix \ref{chap:AppendixCode} gives some details on how the software is implemented, together with a list of some useful libraries used that surely can help to develop a control architecture in the C++ programming language.\\

For some on-going progresses in this scenario, it can be useful for the reader to follow the TWINBOT project [\cite{TWINBOT2019}]. This thesis' context derives from the scenario of this project, but it evolves independently because at the time this thesis was being developed, TWINBOT was in a very early stage.


\section{Future Works}
Since the novelty of the application, further works can be pursued in various directions.\\
For what concerns the experiments, a dynamic simulation, along with a dynamic controller, can be introduced to better analyse the methods adopted. This would mean to include effects that would increase the realism of the simulation, such as buoyancy, thrusters modelling, disturbances of the arm to the vehicle and vice-versa, real tool-grasping effects, even some water currents. Some work has been done in this direction but then it has not been pursued due to the lack of time. These efforts, even if they are not presented in this thesis, showed that the first step to introduce dynamics could be using the plugin \href{https://github.com/freefloating-gazebo/freefloating_gazebo}{FreeFloatingGazebo} [\cite{freeFloatingGazebo}], mentioned in section \ref{sec:simulators}. This one is the most suitable tool to be used from the actual work because its scope is to solve the lack of dynamics of UWSim, expanding the functionalities of the simulator. So, it would be easy to adapt the code to the new simulations. For example, the scene (i.e. the file which describes the simulated scenario) would be the same, and ROS would be always used as interface.\\

Regarding the actual chosen architecture, the Force Torque objective idea can be improved. For example, we can consider a different task reference, calculated not only as a proportional error between the desired force (that is zero) and the actual detected one.\\
Another improvement can be for the Change Goal routine. We could let the forces and the torques modify also the orientation of the goal frame, to reduce/eliminate the angular error between the real goal frame and the one estimated.\\

Regarding the insertion phase, additional problems can be explored. This thesis focused only on collisions that may happen when the peg is already inside the hole. If some contact between the external surface of the hole's structure and the peg's tip happens, the mission fails (i.e., it occurs a stalemate where the peg bounces forever against the hole's surface, unable to find the hole). In the literature, various methods have been explored toward this point. For example, researchers have considered the cases when the peg meets the hole with a bad alignment that creates a two or three points contact (as briefly explained in \ref{sec:artPeg}). However, to the best of this author's knowledge, the \textit{peg-in-hole} problem has never been studied when the protagonists are two autonomous mobile manipulator (in any scenarios, not specifically underwater ones). So, it can be interesting to adapt old tools to this particular (cooperative and underwater) field.\\

Towards a more realistic intervention missions, efforts can be spent to consider ways to localize the robot under the water's surface. In this thesis, it is assumed that all agents have a reference frame in common, but, usually, underwater localization is really an issue. Some cooperative methods (this time not at kinematic level) can be considered to make some surface vessels help the underwater agents to localize themselves (a problem explored in the WiMUST project [\cite{wimust}]).\\

In this thesis, some assumptions have been made for the Vision part. Further works could consider to relax some of them, for example to increase the difficulty of the detection and tracking phases. In an underwater scenario, not always the water permits to watch from afar, and illumination and distortions can be other important issues.\\

There is always a lot of work towards increasing the capacity of \mbox{intervention-AUVs}. For example, we can consider more specifically communication issues and, so, new techniques to exchange data between agents; especially in an underwater scenario, we can't share too much information among robots and with too high frequency. Also, other kinds of assembly problems can be addressed: for instance, a \textit{peg-in-hole} one where the \textit{peg} is held by only one robot and the \textit{hole} by another one.\\ Some objectives of the TWINBOT project [\cite{TWINBOT2019}] aim to study these two just mentioned problems.



\appendix
%%\textbf{}%%%%%%%%%%%%%%%%%%%%%%%%%%%%%%%%%%%%%%%%%%%%%%%%%%%%%%%%%%%%%%%%%%%%%%%%%%%%%%
%2345678901234567890123456789012345678901234567890123456789012345678901234567890
%        1         2         3         4         5         6         7         8
% THESIS CHAPTER


\chapter{C++ Code Scheme}
\label{chap:AppendixCode}
\ifpdf
\graphicspath{{Appendix/Figures/PNG/}{Appendix/Figures/PDF/}{Appendix/Figures/}}
\else
\graphicspath{{Appendix/Figures/EPS/}{Appendix/Figures/}}
\fi

Some effort has been spent to implement a flexible and modular C++ code architecture. The main focus is directed to facilitate the use of the TPIK approach. With this scheme, adding and removing tasks from the code is easy and straightforward. Furthermore, even if ROS is used as interface, other communication methods (for different simulators, or even for real robots) can be used easily, changing a little part of the code. This is due to the fact that all the ROS parts (the \textit{interface} classes) are written in separate files from the ones of the main blocks.\\
Only a rough idea about the code scheme is given here, without too much details. Further explanations can be found in the github page (\url{https://github.com/torydebra/AUV-Coop-Assembly}) and in the code documentation (\url{https://torydebra.github.io/AUV-Coop-Assembly/}).

\section{Tools}
The Control Architecture is implemented in C++ language. Some external support libraries have been used, and, in this section, the most relevant ones are described. Please note that a particular section (section \ref{sec:simulators}) is dedicated to the choice of the simulator, and another lists the main tools used by Vision (section \ref{sec:visionTools}).

\begin{itemize}
	\item \href{http://www.ros.org/}{\textbf{ROS}} (Robot Operating System), the well-known robotic middleware. It is used to communicate with the simulator, which means sending commands to robots and receiving information by the on-going simulations (e.g. robots states, data from sensors, streaming images from cameras).
	
	\item \href{http://eigen.tuxfamily.org/index.php?title=Main_Page}{\textbf{Eigen}} [\cite{eigen}], a C++ library for linear algebra. It is very useful to deal with matrix computation and management in any C++ software.
	
	\item \textbf{CMAT}, another C++ library, developed at GRAAL laboratory of University of Genova (\url{http://www.graal.dibris.unige.it/}). It implements the core functions for the TPIK method, detailed in \cite{IntroMaris1}.
	
	\item \href{http://www.orocos.org/kdl}{\textbf{Orocos KDL}}, a package to deal with kinematic and dynamic chains. Here it is used to compute the Jacobian of the robots, given the arm and the vehicle configuration.
\end{itemize}

\section{The Single Robot Code Scheme}
\begin{figure}[H]
	\begin{center}
		\includegraphics[width=1\columnwidth]{CodeScheme_single.png}
		\caption[C++ code scheme for the single robot]{The C++ code scheme that shows the relationship among the main blocks of the single robot. Parallelepipeds represent the principal blocks, while rectangles contain useful functions used by them. Arrows indicate the communication between the main blocks.}
		\label{fig:codeSchemeSingle}
	\end{center}
\end{figure}

In this section it is presented the scheme for the single carrying robot. The two cooperative robots share the same code, and they are differentiated (when necessary) thanks to their names (\textit{g500\_A} and \textit{g500\_B}).

\begin{itemize}
	\item \textbf{RosInterface}. This class is used to communicate with the simulator, e.g. sending commands to the robot, receiving state and sensor information, and so on. It is obviously ROS-dependent, and it should be replaced if another middleware is used.
	
	\item \textbf{Mission Manager}. This block is the \enquote{main}. It initializes all the useful classes, it creates the tasks list, and it manages the whole mission.
	
	\item \textbf{Controller}. This class is the core of the control; in practice it is the kinematic layer. It generates commands for the vehicle according to the prioritized list of tasks. It is where the iCAT algorithm based on the TPIK approach (section \ref{sec:tpik}) is used.
	
	\item \textbf{Task}. This is an \textit{abstract} class. It acts as a base class for all the specific \textit{concrete} classes of tasks. In this way, the controller and the mission manager simply handle a vector of pointer to Task. The Mission Manager creates a concrete class for each task and then it fills the vector. This vector is the prioritized list of the TPIK method. The controller can iterate the element of this vector and can call the abstract methods of Task without worrying of which real concrete task is actually inside the list.
	
	\item \textbf{Support}. It is a group of various \textit{namespaces}, used for conversions, to print to file, and to use some mathematical formulas.
	
	\item \textbf{Helper}. It contains a group of classes and headers used to help the control scheme (e.g. to compute the Jacobian with KDL) and to log results.
\end{itemize}


\section{The Whole Code Scheme}
\begin{figure}[H]
	\begin{center}
		\includegraphics[width=1\columnwidth]{CodeScheme.png}
		\caption[C++ code scheme for the whole architecture]{The C++ code scheme that shows the relationship among the main blocks of the three robots. On the left and on the right side there is a zoomed out view of the previous figure \ref{fig:codeSchemeSingle}, corresponding to the scheme for the two carrying agents. In the centre, there are the blocks for the Vision robot and for the Coordinator.}
		\label{fig:codeSchemeWhole}
	\end{center}
\end{figure}
	
The two robots must communicate between them and with the Vision robot. Like explained before, ROS is used to communicate with the simulator, but it is also used to make different nodes (e.g. Coordinator and Robots) to communicate.\\
Please note that the Coordinator is not a physical agent: it can be put as a software routine on one robot. This would help with the communication issues typical of underwater scenarios, because only data-exchange between the Coordinator and the other robot will pass through the water.\\
The scheme for the Vision robot is simple because it is driven as a ROV: it is not autonomous and so no TPIK is implemented for it. Anyway, it can be switched easily into an autonomous robot, with or without TPIK (that it is not really necessary for this agent).\\
The Coordinator is a node in charge of dealing with the coordination policy explained in section \ref{sec:coopScheme}. It also needs information from the world (i.e. the simulation) to compute the cooperative velocity.

%%\textbf{}%%%%%%%%%%%%%%%%%%%%%%%%%%%%%%%%%%%%%%%%%%%%%%%%%%%%%%%%%%%%%%%%%%%%%%%%%%%%%%
%2345678901234567890123456789012345678901234567890123456789012345678901234567890
%        1         2         3         4         5         6         7         8
% THESIS CHAPTER


\chapter{Other Algorithms for Object Detection}
\label{chap:AppendixVision}
\ifpdf
    \graphicspath{{Vision/Figures/PNG/}{Vision/Figures/PDF/}{Vision/Figures/}}
\else
    \graphicspath{{Vision/Figures/EPS/}{Vision/Figures/}}
\fi

During the simulations, several trials have been done to find a suitable algorithm for the detection of the hole structure. In Chapter \ref{chap:vision} two methods have been discussed as the successful ones. In this appendix, others that have been discarded are briefly presented. Even if they are not used in the last versions of experiments, they can be useful for other purposes, such as detection of other kind of shapes.\\
Each algorithm is taken from OpenCV Detection tutorials (\url{https://docs.opencv.org/3.4/d9/d97/tutorial_table_of_content_features2d.html}), where also other interesting methods can be found.

\section{Corner Detection with the Shi-Harris method}

\begin{figure}[H]
	\centering
	\includegraphics[width=8.0cm]{goodFeatToTrack}
	\caption[Result of \textit{goodFeaturesToTrack()}]{\textit{goodFeaturesToTrack()} result. Only two detected points are the real corners, and the upper ones are not detected.} 
	\label{fig:goodFeatToTrack}
\end{figure}


Following the tutorial (\url{https://docs.opencv.org/3.4.6/d8/dd8/tutorial_good_features_to_track.html}), it has been implemented a corner detector with the Shi-Harris method [\cite{Shi94goodfeatures}] using the OpenCV function \href{https://docs.opencv.org/3.4.6/dd/d1a/group__imgproc__feature.html#ga1d6bb77486c8f92d79c8793ad995d541}{goodFeaturesToTrack()}.\\
This function, in our case, can be useful to find the corners of the hole structure, which is the necessary initialization for the Tracking method used after (section \ref{sec:visDetect}).\\

The original example lets change the number of maximum points to be found. This is useful to reduce the number of false positive corners. The main problem is that the real corners of the square are not the "best" ones. So we can't simply put this parameter equal to 4. On the other side, with a large number of points, would be then difficult to discriminate the right corners from the others.\\
Other interesting parameters are:
\begin{itemize}
	\item \textbf{minDistance}. The minimum distance among the corners to be found.
	\item \textbf{qualityLevel}. A parameter which characterizes the minimum accepted quality of image corners.
	\item \textbf{blockSize}. Size of an average block for computing a derivative covariance matrix over each pixel's neighbourhood.
	\item \textbf{mask}. To specify a certain region of interest in the image. In such a way, corners are searched only in this region. The problem in our case is that without prior works we can't know where the interesting region is (i.e. the region which contains the hole).
\end{itemize}
The points detected are effectively good feature points (as can be seen in figure \ref{fig:goodFeatToTrack}). But, the best ones are not the ones that we want to detect (i.e., the corner of the square).\\
This method should be used as a low level algorithm, to then help higher level ones. For example, to construct some polygons and to check if these polygons are square/rectangles. However, to follow this direction should be better to start from the edges (as done in section \ref{subsec:findSquare}).\\
  
\section{Canny Edge and Hough Transform}
\label{sec:HoughTrasf}
%https://docs.opencv.org/3.4/d9/db0/tutorial_hough_lines.html

\begin{figure}[H]
	\centering
	\centerline{
		\includegraphics[width=7.0cm]{canny_HoughStandard}
		\qquad
		\includegraphics[width=7.0cm]{canny_HoughProb}
	}
	\caption[Results of the Standard Hough Transform and the Probabilistic one]{Results of the Hough Standard Transform (left) and the Probabilistic one (right). In white they are depicted all the edges detected with Canny; the red lines are the detected straight lines, outputs of the method.}
	\label{fig:HoughStandard}
\end{figure}

The Hough Transform [\cite{DudaHoughTrasf}] is a method to detect straight lines in an image. Usually, a preprocessing of the image with an edge detector is used to improve the results, for example with a Canny Edge Detector [\cite{CannyEdge}].\\

The OpenCV tutorial (\url{https://docs.opencv.org/3.4/d9/db0/tutorial_hough_lines.html}) makes use of two types of Hough Transform: the standard \href{https://docs.opencv.org/3.4/dd/d1a/group__imgproc__feature.html#ga46b4e588934f6c8dfd509cc6e0e4545a}{\textit{HoughLines()}} and the probabilistic \href{https://docs.opencv.org/3.4/dd/d1a/group__imgproc__feature.html#ga8618180a5948286384e3b7ca02f6feeb}{\textit{HoughLinesP()}} [\cite{houghprob}].\\
Results are visible in figure \ref{fig:HoughStandard}. The results on the right shows that the probabilistic version is good to detect the square structure of the hole. 
Thus, this method can be used as a good preliminary step to then extract the corner from the detected square.

\newpage
\section{Bounding Box Detection}
\label{sec:boundingBox}
%https://docs.opencv.org/3.4.6/de/d62/tutorial_bounding_rotated_ellipses.html	

\begin{figure}[H]
	\centering
	\centerline{
	\includegraphics[width=7.0cm]{BoundBox_sourceOnlyPolig}
	\qquad
	\includegraphics[width=7.0cm]{BoundBox_resultOnlyPolig}
	}
	\caption[Result of \emph{findContours()}]{Result of \emph{findContours()}: on the left the original image, on the right the contours detected.}
	\label{fig:BoundBoxresultOnlyPolig}
\end{figure}

\begin{figure}[H]
	\centering
	\includegraphics[width=7.0cm]{BoundBox_resultOnlyRect}
	\caption[Result of Bounding Box detection]{Result of Bounding Box detection, where they are visible the drawn bounding boxes around the holes.}
	\label{fig:BoundBoxresultOnlyRect}
\end{figure}

The code derived from an OpenCV tutorial (\url{https://docs.opencv.org/3.4.6/de/d62/tutorial_bounding_rotated_ellipses.html}).\\
First, a Canny edge detector is used to preprocess the image. Then, the function \href{https://docs.opencv.org/3.4.6/d3/dc0/group__imgproc__shape.html#ga17ed9f5d79ae97bd4c7cf18403e1689a}{\textit{findContours()}} is called to retrieve contours with the algorithm described in \cite{findcountors}. As can be noticed, these initial steps are the same of the implemented method of section \ref{subsec:findSquare}. The difference in this tutorial is that, after these passages, bounding boxes are drawn around some particular shapes detected.\\

The result after the first step is shown in figure \ref{fig:BoundBoxresultOnlyPolig}. We can see that this passage already reveals the square, that is important for the method described in \mbox{section \ref{subsec:findSquare}.}\\
Instead, the algorithm presented here continues in a different direction, which brings us to the final result of figure \ref{fig:BoundBoxresultOnlyRect}.\\

This tutorial is recalled because it can be useful to find other kinds of shapes, for example an hole structure which is not a rectangle or a square.


\section{2D Feature Matching \& Homography}
%https://docs.opencv.org/3.4/d7/dff/tutorial_feature_homography.html
\begin{figure}[H]
	\centering
	\centerline{
		\includegraphics[width=3.8cm]{new_featHomog_SURF_templKeyPoint}
		\qquad
		\includegraphics[width=7.1cm]{new_featHomog_SURF_cameraKeyPoint}
	}
	\vspace{10px}
	\includegraphics[width=9cm]{new_featHomog_SURF_result}
	\caption[Result of 2D Feature Matching]{Result of the 2D Feature Matching. Above, the blue circles are the detected features in the \textit{object image} (on the left) and in the \textit{scene image} (on the right). Below, the output of the matching passage, which shows clearly a bad outcome.}
	\label{fig:featHomog}
\end{figure}

\textit{Image features} are small patches that are useful to compute similarities between images. These are different from corner points; they indicates particular details that stand out in the image. Detecting these areas is useful to recognize objects of interest, as a sort of \textit{template matching} (please note that this method is not a template matching as the one of section \ref{subsec:templateMatch}).\\
The \textit{descriptors} of these features contain the visual description of the patches, which are used to recognize similarities between different images.\\

This method needs an \textit{object image} and a \textit{scene image}. The first is a sort of template which contains only the object to be found (in this case, the square face of the hole). The second is the image in which we want to detect this object (in this case, what the camera is seeing).\\
After good features are extracted from both images, the \textit{descriptors} are used to \textit{match} them, thus, detecting the object in the scene.
Then, it is necessary to find the perspective transformation between the object image and the scene (i.e. find the homography). This is needed to take into account that usually the pose and the scaling of the object in the scene are not the same of the ones of the object image.\\
The OpenCV tutorial (\url{https://docs.opencv.org/3.4/d7/dff/tutorial_feature_homography.html}) uses different tools:
\begin{itemize}
	\item \textbf{SURF} (Speeded Up Robust Features) Detector [\cite{surfDet}] to extract the features, and to compute the descriptors.
	\item \textbf{FLANN} (Fast Library for Approximate Nearest Neighbors) matcher [\cite{flannMatch}] to match the features of the two images.
	\item \textbf{Lowe's ratio test} [\cite{loweTest}] to take only the best matches.
	\item \textbf{RANSAC} (RANdom SAmple Consensus) [\cite{ransacHomog}] method to find the homography with the function \href{https://docs.opencv.org/3.4/d9/d0c/group__calib3d.html#ga4abc2ece9fab9398f2e560d53c8c9780}{\textit{findHomography()}}.
\end{itemize}

\vspace{20px}
In this case, results are unsatisfactory, as can be seen in figure \ref{fig:featHomog}. The main problem is that in this particular scene there are not nice distinct features. Also, the symmetry of the structure does not help, because there are a lot of particulars that are similar (like the square sides and the holes). As can be seen in the results of the previously cited \href{https://docs.opencv.org/3.4/d7/dff/tutorial_feature_homography.html}{tutorial}, good outcomes are obtained for food boxes. In fact, this method is often exploited for scenes where a lot of details are present (e.g. graffiti painting, supermarket shelf). In underwater cases, realistic infrastructures have not so many details, and so also the simulated scenario of this thesis.\\

There are a lot of parameters to set for the three main tools (SURF, FLANN, and RANSAC). Various options have been tried but no-one was satisfactory. Also, different detectors (like SWIFT [\cite{loweTest}]) and matchers (like Brute-Force), have been tried, again with poor results.\\
Anyway, the variety of tools and parameters makes this method suitable for a lot of applications, and it should be taken into consideration in other applications.








\bibliographystyle{Classes/RoboticsBiblio}    % bibliography style
\renewcommand{\bibname}{References}           % change default name Bibliography to References
\bibliography{References/references}          % References file
\addcontentsline{toc}{chapter}{References}    % add References to contents page

\end{document}




%TODO Unmanned non vuol dire quello che pensi, rileggi state of art per refusi
%TODO chiedi se devo linkare codice mio... tipo Detector di che è classe per i seguenti algo di visioni che sto per elencare